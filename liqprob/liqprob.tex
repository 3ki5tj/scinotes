\documentclass{book}
\usepackage{indentfirst}
\usepackage{amsmath}
\usepackage{amsthm}
\usepackage{mathrsfs}
\usepackage{bm}
\usepackage{upgreek}
\usepackage{hyperref}
\usepackage{cite}
%\usepackage{relsize}
\begin{document}


\newcommand{\problem}[1]{\subsection{#1}}
\newcommand{\hint}[1]{[\,\emph{Hint.} #1]}

\newcommand{\vct}[1]{\boldsymbol{\mathbf{#1}}}
\newcommand{\vr}{\vct{r}}
\newcommand{\vrN}{\vr^N}
\newcommand{\vrn}{\vr^n}
\newcommand{\dvr}{\frac{ d \vr  }{(2\pi)^3}}
\newcommand{\vx}{\vct{x}}
\newcommand{\vxN}{\vx^N}
\newcommand{\vxn}{\vx^n}
\newcommand{\dvx}{\frac{ d \vx  }{(2\pi)^3}}
\newcommand{\vk}{\vct{k}}
\newcommand{\dvk}{\frac{ d \vk  }{(2\pi)^3}}
% Fourier transform
\newcommand{\FT}[1]{\tilde{#1}}
\newcommand{\FTc}{\FT{c}}
\newcommand{\FTh}{\FT{h}}

% add a superscript ``ex''
\newcommand{\supex}[1]{ { { #1 }^{ \mathrm{ex} } } }
\newcommand{\Pex}{\supex{P}}
\newcommand{\Fex}{\supex{F}}
\newcommand{\muex}{\supex{\mu}}
\newcommand{\kex}{\supex{\kappa}}
\newcommand{\Chn}{\mathscr{C}}
%\newcommand{\Chn}{\mathcal{C}}
%\newcommand{\Chn}{\mathsf{C}}
\newcommand{\secref}[1]{Sec. \ref{#1}}

\newcommand{\llbra}{[\![}
\newcommand{\llket}{]\!]}

\newcommand{\plam}{\partial_\lambda}
\newcommand{\pbet}{\partial_\beta}



\title{Challenging Problems Liquid Theory \\
  (For the Pettitt Group)}
\date{}
\maketitle

\tableofcontents





\chapter{Basic mathematical tools}



\section{Gamma function}



\problem{Gamma (factorial) function}



1. The gamma or factorial function is defined as
\begin{equation}
  (z - 1)! = \Gamma(z)
\equiv
  \int_0^\infty t^{z-1} \, \exp(-t) \, d t,
  \label{eq:gammaDef}
\end{equation}
%
By partial integration, show that
\begin{equation}
  \Gamma(z + 1) = z \, \Gamma(z).
  \label{eq:gammaRecur}
\end{equation}

\hint{
$\Gamma(z + 1)
=
  -\int_0^\infty t^z \, d \exp(-t)
=
  \int_0^\infty \exp(-t) \, d (t^z)$.
}

2. If $z$ is an integer, show that
\[
  \Gamma(z) = 1 \cdot 2 \cdots (z - 1),
\]
which agrees with the definition of factorial $(z-1)!$.



\problem{Product formula of the gamma function}



It can be shown\cite{whittaker} that the $\Gamma(z)$
is the $n\rightarrow \infty$ limit of
\begin{equation}
  \Pi(z, n)
=
  \int_0^n
  \left(
    1 - \frac t n
  \right)^n \,
  t^{z-1} \,
  d t.
\label{eq:gammaPi}
\end{equation}
The basic argument is
$\lim_{n \rightarrow \infty} (1 - t/n)^n = \exp(-t)$.

By partial integration, show that
\[
  \Pi(z, n)
=
  \frac{ 1 \cdot 2 \cdots n }
  { z (z + 1) \cdots (z + n) }
  \, n^z.
\]
The $n \rightarrow \infty$ limit is
the Euler's infinite product formula of the gamma function:
\begin{equation}
  \frac 1 {z\,!}
=
  \frac 1 {\Gamma(z + 1)}
=
  \frac 1 {z \, \Gamma(z)}
=
  \lim_{n \rightarrow \infty}
  \left( 1 + \frac z 1 \right)
  \left( 1 + \frac z 2 \right)
  \cdots
  \left( 1 + \frac z n \right)
  n^{-z}.
  \label{eq:gammaEulerProd}
\end{equation}



\problem{Gamma and sine functions}



1. Similar to Eq. \eqref{eq:gammaEulerProd},
we can develop an infinite product of the sine function.
%
The zeros of $\sin(\pi z)$ are $0, \pm 1, \pm 2, \dots$
%
From this, show that
\begin{equation}
  \frac { \sin( \pi z ) } { \pi z }
=
  \lim_{n \rightarrow \infty}
  \left( 1 - \frac {z^2} {1^2} \right)
  \left( 1 - \frac {z^2} {2^2} \right)
  \cdots
  \left( 1 - \frac {z^2} {n^2} \right)
  \label{eq:sin_infprod}
\end{equation}

2. By a comparison to Eq. \eqref{eq:gammaEulerProd},
show that
\begin{equation}
  \frac 1 {z! \, (-z)!}
=
  \frac 1 {\Gamma(1+z)! \, \Gamma(1-z)!}
=
  \frac { \sin( \pi z ) } { \pi z }.
  \label{eq:gammasin}
\end{equation}
%
Using Eq. \eqref{eq:gammaRecur}, show that
\begin{equation}
  \frac 1 {\Gamma(z)! \, \Gamma(1-z)!}
=
  \frac { \pi } { \sin( \pi z ) }.
  \label{eq:gammasin}
\end{equation}

3. Using $z = 1/2$, show that
\begin{equation}
  \left( -\frac 1 2 \right)!
=
  \Gamma\left( \frac 1 2 \right)
=
  \sqrt \pi.
  \label{eq:gammahalf}
\end{equation}



\problem{Doubling formula}



1. Using Eq. \eqref{eq:gammaEulerProd},
show that\cite{whittaker}
\[
  \phi(z)
\equiv
  \frac{
    \left( 2z \right)!
  }
  {
    2^{2z} \, z! \, \left( z - \frac 1 2 \right)!
  }
\]
is independent of $z$.

\hint{
  For the expansion of $(2z)!$,
  the $n$ in Eq. \eqref{eq:gammaEulerProd} should be extended to $2n$,
  twice as that for the expansions of $z!$ and $(z-\frac 1 2)$.
}

2. By using $z = 1, 2, 3, \dots$, show that, we always have
\[
  \phi(z) = \frac 1 {\sqrt \pi}.
\]

3. Prove the doubling formula\cite{whittaker}
\begin{equation}
  \left( z - \frac 1 2 \right)!
=
  \frac{ (2z)! }
  { 2^{2z} \, z! }
  \sqrt \pi
  \label{eq:doubling}
\end{equation}
Thus for an integer $n$,
we have
\begin{align}
  \left(n - \frac 1 2 \right)!
&=
  \frac{ (2 n - 1)!! }
  { 2^n } \,
  \sqrt \pi.
  \label{eq:doubling_odd}
\\
  n!
&=
  \frac{ (2 n)!! }
  { 2^n }.
  \label{eq:doubling_even}
\end{align}
Here
\begin{equation}
  k!!
\equiv
\begin{cases}
  1 \cdot 3 \cdot \cdots \cdot k  & \mbox{if $k$ is odd,} \\
  2 \cdot 4 \cdot \cdots \cdot k  & \mbox{if $k$ is even.}
\end{cases}
\end{equation}
This formula is useful in getting rid of the factorial of a half integer.




\section{Beta function}



\problem{Integral of powers of the sine function}



1. Show that
\begin{align}
  I_n
  \equiv \int_0^\pi \sin^n \theta \, d\theta
  =
  \begin{cases}
    2   \, (n-1)!! / n!!    &   \mbox{if $n$ is odd,} \\
    \pi \, (n-1)!! / n!!    &   \mbox{if $n$ is even.} \\
  \end{cases}
  \label{eq:intsinn}
\end{align}
\hint{
By partial integration:
\begin{align*}
  I_n
  &= - \int_0^\pi \sin^{n-1} \theta \, d\cos\theta
  = (n - 1) \, \int_0^\pi \sin^{n-2} \theta \, \cos^2\theta \, d\theta
  \\
  &= (n - 1) I_{n-2} - (n - 1) I_n,
\end{align*}
where we have used $\cos^2 \theta = 1 - \sin^2 \theta$ in the last step.
%
Thus,
\[
  I_n = \frac{ n - 1} n \, I_{n-2}.
\]
Apply the above formula repeatedly to show \eqref{eq:intsinn}.
}

2. Using Eqs. \eqref{eq:doubling_odd} and \eqref{eq:doubling_even},
show that
\begin{equation}
  I_n
=
  \frac {
    \left( \dfrac{n-1} 2 \right)!
  }
  {
    \left( \dfrac n 2 \right)!
  }
  \sqrt\pi.
  \label{eq:intsinnb}
\end{equation}




\problem{Integral of powers of the sine and cosine functions}



1. Show that for an even $m$
\begin{align}
  I_{m, n}
& \equiv
  2 \int_0^{\pi/2} \cos^m \theta \, \sin^n \theta \, d\theta
=
  \frac{(m-1)!! \, n!!} { (m+n)!! } I_{n},
  \label{eq:intcosmsinn}
\end{align}
where $I_{n} = I_{0, n}$
is the integral given by Eq. \eqref{eq:intsinn} or \eqref{eq:intsinnb}.

\hint{
By partial integration, we have
\begin{align*}
  I_{m, n}
&=
  \frac{1}{n+1} \int_0^\pi \cos^{m-1} \theta \, d \left( \sin^{n+1}\theta \right)
\\
&=
  \frac{m - 1}{n+1} \,
  \int_0^\pi \cos^{m-2}\theta \, \sin^{n+2} \theta \, d\theta
  \\
&=
  \frac{m - 1}{n+1} \,
  \int_0^\pi \cos^{m-2}\theta \, \left(1 - \cos^2 \theta\right) \, \sin^{n} \theta\, d\theta
  \\
&= \frac{m-1}{n + 1} I_{m-2, n} - \frac{m-1}{n + 1} I_{m, n},
\end{align*}
%
Thus,
\[
  I_{m, n} = \frac{m-1}{m + n} I_{m-2, n},
\]
which leads to \eqref{eq:intsinn}.
}

2. Further, show that
\begin{align}
  I_{m, n}
&=
  \left.
    \left( \frac{ m - 1}{2} \right)! \, \left(\frac {n - 1} 2\right)!
  \middle/
    \left( \frac {m + n} 2 \right)!
  \right.
  \label{eq:intcosmsinnb}
\end{align}
by using Eqs. \eqref{eq:doubling_odd} and \eqref{eq:doubling_even},




\problem{Beta function}



We now establish a generalization of Eq. \eqref{eq:intcosmsinn}.

1. Show that
\[
  K = 4 \int_0^\infty \int_0^\infty
  \exp(-x^2 - y^2) \,
  x^{2p+1} \, y^{2q+1} \, dx \, dy
=
  p! \, q!.
\]

\hint{
  Let $X = x^2$ and $Y = y^2$,
  then use the gamma function defined in Eq. \eqref{eq:gammaDef}.
}

2. Show that in the polar coordinates: $x = r \cos\theta$, $y = r\sin\theta$,
we have
\begin{align*}
  K
&=
  4 \int_0^r \exp(-r^2) \, r^{2p+2q+3} \, dr
  \int_0^{\pi/2}
    \cos^{2p+1}\theta \,
    \sin^{2q+1}\theta \, d\theta.
\\
&=
  2 \, (p + q + 1)! \,
  \int_0^{\pi/2}
    \cos^{2p+1}\theta \,
    \sin^{2q+1}\theta \, d\theta.
\end{align*}

3. Show that the beta function
\begin{equation}
  B(p + 1, q + 1)
\equiv
  2 \int_0^{\pi/2}
    \cos^{2p+1}\theta \,
    \sin^{2q+1}\theta \, d\theta
= \frac{ p! \, q! } { (p + q + 1)! },
  \label{eq:betaDef}
\end{equation}
%
This relation,
unlike \eqref{eq:intcosmsinnb},
holds for real (even complex) $p$ and $q$.


4. Verify that Eq. \eqref{eq:betaDef}
agrees with Eq. \eqref{eq:intcosmsinnb}.



\section{Gaussian integral}



\problem{Gaussian integral}



1. Show that
\begin{equation}
  \int_{-\infty}^{\infty}
    \exp\left( -\frac{1}{2} A \, x^2 \right) \, dx
  =
  \sqrt{ \frac{ 2 \pi }{ A } }.
  \label{eq:GaussianIntegral}
\end{equation}

\hint{
  Denote the integral on the left hand side as $I$.
  Show that
  \begin{align*}
    I^2
    &=
    \int_{-\infty}^{\infty}
      \exp\left( -\frac{1}{2} A \, x^2 \right) \, dx
    \int_{-\infty}^{\infty}
      \exp\left( -\frac{1}{2} A \, y^2 \right) \, dy
    \\
    &=
    \int_{0}^{2\pi}
    \int_{0}^{\infty}
      \exp\left( -\frac{1}{2} A \, r^2 \right) \, r \, dr \, d\theta
    \\
    &=
    2\pi
    \int_{-\infty}^{\infty}
    \exp\left[ -A \left( \frac{r^2}{2} \right) \right]
    \, d\left( \frac{r^2}{2} \right),
  \end{align*}
where we have changed from the Cartesian to polar coordinates in the second step.
}

2. Alternatively, change variable $t = A \, x^2/2$,
and evaluate the integral by the gamma function
defined by Eq. \eqref{eq:gammaDef}.
%
Then use Eq. \eqref{eq:gammahalf}.



\problem{Gaussian integral (matrix form)}



1. Show the matrix form of \eqref{eq:GaussianIntegral}
\begin{equation}
  \int_{-\infty}^{\infty}
  \exp\left( -\frac{1}{2} \vct{x}^T \vct{A} \vct{x} \right) \, d\vct{x}
  =
  \frac{ (2\pi)^{n/2} }
  { |\vct A|^{1/2} },
  \label{eq:GaussianIntegralMatrix}
\end{equation}
%
where
$\vct x = \{x_1, \dots, x_n\}^T$,
and
$\vct A$ is an $n$ by $n$ symmetric positive-semidefinite matrix
(i.e., $\vct x^T \vct A \vct x \ge 0$ for any $x$),
and
$|\vct A|$ is the determinant.

\hint{
  First, consider the case that $\vct A$ is an diagonal matrix.
  Show that the integral in \eqref{eq:GaussianIntegralMatrix}
  is equivalent to $n$ one-dimensional integrals,
  and evaluate them by \eqref{eq:GaussianIntegral}.
  Then, for a general $\vct A$, consider
  \[
    \vct x = \vct U \vct y,
  \]
  which $\vct U$ being a orthonormal matrix that diagonalize $\vct A$:
  \[
    \vct A = \vct U^T \vct \Lambda \vct U,
  \]
  where $\vct \Lambda$ is a diagonal matrix.
}

2. What happens if $\vct A$ is not a symmetric matrix?




\problem{Gaussian integral with a bias}



More generally, show that
\begin{equation}
  \int_{-\infty}^{\infty}
  \exp\left(
    -\frac{1}{2} \vct{x}^T \, \vct{A} \, \vct{x}
    + \vct b^T \, \vct x
  \right) \, d\vct{x}
  =
  \frac{ (2\pi)^{n/2} }
  { |\vct A|^{1/2} }
  \exp\left( \frac{1}{2} \vct b^T \vct A^{-1} \vct b \right),
  \label{eq:GaussianIntegralMatrixb}
\end{equation}
%
where $\vct b = \{b_1, \dots, b_n\}^T$.
%
\hint{ Complete the square. }



\problem{Derivatives of Gaussian integrals}



1. By differentiating \eqref{eq:GaussianIntegral} with respect to $A$, show that
\begin{equation}
  \langle x^2 \rangle
\equiv \frac {
  \int_{-\infty}^{\infty}
    x^2 \exp\left( - \frac{1}{2} A\,x^2 \right) \, dx
} {
  \int_{-\infty}^{\infty}
  \exp\left( - \frac{1}{2} A\,x^2 \right) \, dx
}
=
\frac{1}{A}.
\end{equation}

2.
Define the matrix $\vct{c}$, whose elements are defined as
\begin{equation*}
c_{ij}
\equiv \langle x_i \, x_j \rangle
\equiv \frac {
  \int_{-\infty}^{\infty}
    x_i \, x_j \exp\left( -\frac{1}{2} \vct{x}^T \vct{A} \vct{x} \right) \, d\vct{x}
} {
\int_{-\infty}^{\infty}
  \exp\left( -\frac{1}{2} \vct{x}^T \vct{A} \vct{x} \right) \, d\vct{x}
},
\end{equation*}
%
where $\vct A$ is a symmetric matrix.
%
By differentiating \eqref{eq:GaussianIntegralMatrix} with respect to $A_{ij}$, show that
%
\begin{equation}
  \vct{c} = \vct{c}^T = \vct A^{-1},
  \label{eq:GaussianCorrelation}
\end{equation}
%
where
$\vct{A}^{-1}$ is the inverse matrix of $\vct{A}$
(thanks to Justin Drake).

\hint{
Use the identity
\[
  \frac{\partial}{\partial A_{ij}}
  \log |\vct{A}|
=
  \left( \vct{A}^{-1} \right)_{ji},
\]
which can be shown by using the minors of $\vct A$.
}

3. By differentiating \eqref{eq:GaussianIntegralMatrixb} with respect to $b_i$ and $b_j$,
and then setting $\vct b = \vct 0$, show \eqref{eq:GaussianCorrelation}.





\problem{Volume of high-dimensional spheres}



In this problem, we will find the volume of a high-dimensional sphere
by using the Gaussian integrals.

1. Argue that the volume of a $D$-dimensional sphere of radius $R$ is proportional to
$R^D$; and the surface area $S_D(R)$ is proportional to $R^{D-1}$, and
\[
  S_D(R) = S_D(1) \, R^{D-1}.
\]

2. Evaluate the integral
\[
  I_D \equiv \int \exp(-r^2) \, d\vr = \int \exp(-r^2) S_D(r) dr
\]
by the gamma function defined in Eq. \eqref{eq:gammaDef}.

3. Evaluate the same integral as
\[
  I
  =
\left( \int \exp(-x_1^2) \, d x_1 \right)
\dots
\left( \int \exp(-x_D^2) \, d x_D \right).
\]

4. Show that
\begin{align}
  S_D(R) &= \frac{ 2 \, \pi^{D/2} } { \Gamma(D/2) } R^{D-1},
  \label{eq:SD} \\
  V_D(R) &= \frac { \pi^{D/2} } { \Gamma(D/2+1) } R^D.
  \label{eq:VD}
\end{align}
%
Particularly,
\begin{align}
  S_D(R)
=
\begin{cases}
  2 \, (2 \, \pi)^{(D-1)/2} R^{D-1} / (D-2)!!  & \mbox{if $D$ is odd,} \\
       (2 \, \pi)^{D/2} R^{D-1} / (D-2)!!      & \mbox{if $D$ is even,} \\
\end{cases}
%\tag{\ref{eq:SD}$'$}
\label{eq:SDoe}
\end{align}
and
\begin{align}
  V_D(R)
=
\begin{cases}
  2 \, (2 \, \pi)^{(D-1)/2} R^D / D!!  & \mbox{if $D$ is odd,} \\
       (2 \, \pi)^{D/2} R^D / D!!      & \mbox{if $D$ is even.} \\
\end{cases}
%\tag{\ref{eq:VD}$'$}
\label{eq:VDoe}
\end{align}



\problem{High-dimensional spherical coordinates}



In two dimensions, the volume element of spherical coordinates is
\[
  r \, dr \, d\theta_2,
\]
where the integral $\theta_2$ goes from $0$ to $2\pi$.

In three dimensions, it becomes
\[
  r^2 \, dr \,
  \sin \theta_3 \, d \theta_3 \,
  d \theta_2,
\]
where $\theta_3$ ($0 \le \theta_3 \le \pi$)
is the angle between $\vr$ and the $\vct{z} = \vct{x}_3$ axis.

1. Show that the above result in $D$ dimensions is given by
\begin{equation}
  d\vr^{D}
=
  r^{D-1} \, dr \;
  \sin^{D-2} \theta_D \, d \theta_D \,
  \cdots \,
  \sin^{k-2} \theta_k \, d \theta_k \,
  \cdots \,
  \sin \theta_3 \, d \theta_3 \;
  d \theta_2,
  \label{eq:sphrcoordiff}
\end{equation}
where $\theta_k$ is the angle between $\vx_k$ and
the projection of $\vr$ in the subspace
spanned by $\vx_1$, \dots, and $\vx_k$.
Here, $0 \le \theta_k \le \pi$ for $k \ge 3$,
and $0 \le \theta_2 \le 2\pi$.

\hint{
Start with the relation:
\[
  r^{D-1} \, dr \, d\vct\uptheta^{D}
= s^{D-2} \, ds \, d\vct\uptheta^{D-1} \, d x_D,
\]
where
$d\vct\uptheta^k$ denotes the angular element in $k$ dimensions,
and
$s$ is the length of the projection of $\vr$
onto the first $(D-1)$-dimensional subspace.
%
Thus
\begin{align*}
  s   &= r \, \cos \theta_D, \\
  x_D &= r \, \sin \theta_D,
\end{align*}
with the Jacobian $\partial(x_D, s)/\partial(r, \theta_D) = r$.
%
This means that
\[
  ds \, d x_D
=
  r dr \, d \theta_D
\]
It follows that
\begin{align}
  r^{D-1} \, dr \, d\vct\uptheta^{D}
=
  r^{D-1} \, dr \, \sin^{D-2} \theta_D \, d\theta_D \, d\vct\uptheta^{D-1}.
  \label{eq:drD_recursion}
\end{align}
Then, show \eqref{eq:sphrcoordiff} by induction.
}

2. By integrating over the angular variables,
show that for a spherical symmetric function
the surface element is given by \eqref{eq:SD}.

\hint{
From \eqref{eq:drD_recursion}, we have
\begin{align}
    S_D(1)
  &= I_{D-2} \, S_{D-1}(1)
   \label{eq:SD_recursion}
\\
  &= I_{D-2} \, \cdots I_1 \cdot (2 \, \pi),
\notag
\end{align}
where $I_n$ ($1 \le n \le D - 2$)
is defined by Eq. \eqref{eq:intsinn}.
Use, however, Eq. \eqref{eq:intsinnb}.
}



\section{Bessel function}



\problem{Bessel function}



For integral orders, the Bessel function can be generated from
\begin{align*}
  \exp \left[ \frac{z}{2} \left(t - \frac 1 t \right) \right]
&=
  \sum_{\nu = -\infty}^\infty J_\nu(z) \, z^n.
\end{align*}

1. From this definition,
show that the series expansion of the Bessel function
is given by
\begin{align}
  J_\nu(z)
&=
  \sum_{s = 0}^\infty
  (-)^s \, \frac{ (z/2)^{\nu + 2 s} } { s! \, (\nu + s)! }.
  \label{eq:Bessel_series}
\end{align}
Notet that although Eq. \eqref{eq:Bessel_series}
is derived for an integral $\nu$,
it is also valid for a real $\nu$.

\hint{
Multiply the series
\begin{align*}
  \exp \frac{z \, t} {2}
&=
  \sum_{l = 0}^\infty \frac{1}{l!} \left( \frac{z \,t}{2} \right)^l,
\end{align*}
and
\begin{align*}
  \exp \frac{z}{2 \, t}
&=
  \sum_{s = 0}^\infty \frac{1}{s!} \left( \frac{z}{2 \, t} \right)^s.
\end{align*}
Then collect the coefficient of $t^{l - s}$, with $\nu = l - s$.
}


2. Show that
\begin{align}
  \int_0^\pi \exp( - i z \cos \theta) \, \sin^{2\,\nu} \theta \, d\theta
=
  \frac{ \nu! }{ (z/2)^\nu } \, I_{2\nu} \, J_\nu(z),
  \label{eq:intexpsin}
\end{align}
where $I_{2\nu}$ is defined in Eq. \eqref{eq:intsinn}.

\hint{
Expand
\[
  \exp(-iz\cos\theta)
  = \sum_{s = 0}^\infty (-)^s \frac{ (z \cos \theta )^{2s} } { (2s)! },
\]
then evaluate the integral for each term by using Eq. \eqref{eq:intcosmsinnb}.
}



\problem{Fourier transform}



1. The Fourier transform in $D$ dimensions is defined as
\begin{equation}
  \tilde f(\vk)
\equiv
  \int d\vr \, \exp(-i\vk \cdot \vx) \, f(\vx).
  \label{eq:FourierTransform}
\end{equation}
Verify that the inverse transform is given by
\begin{equation}
  f(\vx)
\equiv
  \int \dvk \, \exp(i\vk \cdot \vx) \, \tilde f(\vk).
  \label{eq:inverseFourierTransform}
\end{equation}

\hint{
  Use the identity
\[
  \int d\vk \, \exp[i\vk \cdot (\vx - \vx')]
= (2 \pi)^D \, \delta(\vx - \vx').
\]
where $\delta(\vx - \vx')$ is the delta function
that satisfies the relation
\[
  \int d\vx \, \delta(\vx - \vx') \, f(\vx)
  = f(\vx'),
\]
for an arbitrary function $f(\vx)$.
}

2. By the differentiating Eq. \eqref{eq:inverseFourierTransform},
show that the Fourier transform of $\nabla f(\vx)$ is given by $i\vk \, \tilde f(\vk)$.

3. Obtain the Fourier transform of the Gaussian function
\[
  f(\vx) \equiv \exp\left( - \frac{1}{2} \vct x^T \vct A \vct x \right)
\]
by setting $\vct b = -i \vct k$ in \eqref{eq:GaussianIntegralMatrixb}.
%
Also, verify the inverse transform
is indeed given by Eq. \eqref{eq:inverseFourierTransform}.




\problem{Convolution}

1. Show that the Fourier transform of the convolution
\begin{equation}
  (a*b)(\vx)
\equiv
  \int a(\vx - \vx') \, b(\vx') \, d\vx'
  \label{eq:convolution}
\end{equation}
is given by $\tilde{a}(\vk) \, \tilde{b}(\vk)$.

2. Show that
\begin{equation}
  \int a(\vx) \, b(\vx) \, d\vx
=
  \int \tilde{a}(\vk) \, b(-\vk) \, \dvk.
\end{equation}




\problem{3D Fourier transform of a spherical function}



1. Show that in three dimensions,
the Fourier transform of spherical-symmetric function
$f(\vr) = f(r)$, with $r = |\vr|$,
is also spherical symmetric.
%
That is, $\tilde f(\vk) = \tilde f(k)$, with $k = |\vk|$.

2. Further, show that
\begin{equation}
  \tilde f(k)
=
\frac{ 4 \, \pi } {k}
  \int
  f(r) \,
  \sin(k\,r) \,
  r \,
  dr.
  \label{eq:FourierTransformSpherical3D}
\end{equation}

2. Show that in three dimensions,
the convolution of
two spherical-symmetric functions $a(r)$ and $b(r)$
satisfies\cite{hill} (page 204)
\[
  \int a(\vr - \vr') \, b(\vr') \, d\vr'
=
  \frac{2\pi}{r}
  \int_0^\infty a(s) \, s \, ds
  \int_{|r-s|}^{r+s} b(t) \, t \, dt.
\]



\problem{General Fourier transform of a spherical function}



1. Show that, in $D$ dimensions
\begin{equation}
  \tilde f(k)
=
  \frac{(2 \pi)^{D/2}} {k^{D/2 - 1}}
  \int
  f(r) \,
  J_{D/2-1}(k\,r) \,
  r^{D/2} \,
  dr.
  \label{eq:FourierTransformSpherical}
\end{equation}
Here, $J_\nu(z)$ is the Bessel function
given by Eq. \eqref{eq:Bessel_series}.

\hint{
By using Eq. \eqref{eq:SD_recursion},
we have
\begin{align*}
  \tilde f(k)
=
  \int_0^\infty
  \int_0^\pi \exp(-i \, k \, r \, \cos \theta_D)
  \, \sin^{D-2} \theta_D \, d\sin \theta_D
  \, S_{D-1}(1)
  \, r^{D-1} \, dr,
\end{align*}
where $S_{D-1}(1)$ results from integrating
$\theta_{D-1}$, \dots, and $\theta_2$.

Then, use Eq. \eqref{eq:intexpsin} with $\nu = D/2-1$ to show that
\begin{align*}
  \tilde f(k)
=
  \int_0^\infty
  \frac{ J_{D/2-1}(kr) }
  { (kr/2)^{D/2-1} }
  (D/2-1)! \,
  I_{D-2} \,
  S_{D-1}(1) \,
  r^{D-1} \, dr.
\end{align*}
Then simplify the expression using Eqs. \eqref{eq:SD_recursion} and \eqref{eq:SD}.
}

2. Show that in the $D = 3$ case,
Eq. \eqref{eq:FourierTransformSpherical}
is reduced to
Eq. \eqref{eq:FourierTransformSpherical3D}.




\chapter{Thermodynamics and statistical mechanics}



\section{Virial theorem}



\problem{Basic virial theorem}

1. Show that, for a liquid with the pair potential being $u(r)$,
the pressure is given by
\begin{equation}
  \beta \, P = \rho - \frac{\beta \rho^2} {2\, D} \int r \, u'(r) \, g(r) \, d\vr,
\end{equation}
where $\beta = 1/(k_BT)$, and $D$ is the dimension.

2. Now consider a liquid with the discontinuous square well potential
in three dimensions $D = 3$,
\begin{equation}
  u(r) =
  \begin{cases}
    \infty      \quad   & r < 1,          \\
    -\epsilon           & 1 \le r < a,    \\
    0                   & a \le r,
  \end{cases}
\end{equation}
where $a > 1$.
%
Show that the pressure, $P$, is given by
\begin{equation}
  \beta \, P = \rho
            + \frac{2 \pi \rho^2}{3}
              \big[g(1^+)
              + (e^{\beta \epsilon} - 1) a^3 g(a^-)\big].
\end{equation}
%
The hard-sphere limit is obtained with $\epsilon \rightarrow 0$
or $a \rightarrow \infty$.

\hint{
Express $u'(r)$ by $e'(r)$, where $e(r) = e^{-\beta u(r)}$;
and use the fact that $y(r) = g(r)/e(r)$ is always a continuous function of $r$.
Cf. \cite{hansen} Sec. 2.5, Eq. (2.5.22)-(2.5.26).
}


\problem{Virial theorem for cluster integrals}

Show that, in $D$ dimensions,
\begin{align*}
  &
  3 \, D
  \int
  f(\vr_{12}) f(\vr_{23}) f(\vr_{34}) f(\vr_{41})
  \, d\vr_2 \, d\vr_3 \, d\vr_4 \\
  &=
  -4 \int
  \vr_{12} \cdot [\nabla_1 f(\vr_{12})]
  \, f(\vr_{23}) \, f(\vr_{34}) \, f(\vr_{41})
  \, d\vr_2 \, d\vr_3 \, d\vr_4.
\end{align*}

\hint{
Show that the integral
\[
  I(s)
= s^{3D}
  \int
  f(s \, \vr_{12}) \,
  f(s \, \vr_{23}) \,
  f(s \, \vr_{34}) \,
  f(s \, \vr_{41}) \,
  d\vr_2 \, d\vr_3 \, d\vr_4,
\]
is independent of $s$.
Then, differentiate $I(s)$ with respect to $s$.
}



\section{Equation of state}



\problem{Pressure and chemical potential, I}



1. Show that pressure $P$ and chemical potential $\mu$ are related by
\begin{equation}
  \left(
    \frac{ \partial P } { \partial \rho }
  \right)_T
=
  \rho
  \left(
    \frac{ \partial \mu } { \partial \rho }
  \right)_T.
  \label{eq:dPdrho}
\end{equation}

\hint{
From $d(E - TS) = -S \, dT - P \, dV + \mu \, dN$, show that
\[
  -\left(
    \frac{ \partial P }{ \partial N }
  \right)_{T, \, V}
= \left(
    \frac{ \partial \mu }{ \partial V }
  \right)_{T, \, N}.
\]
Then use the fact that $P$ depends on $N$ and $V$ only through $\rho = N/V$.
}


2. Using Eq. \eqref{eq:dPdrho}, show that the density series
\begin{align}
  \beta \, P
&=
  \sum_{n = 1}^\infty B_n \, \rho^n,
  \label{eq:series_Prho}
\end{align}
and
\begin{align}
  \ln z = \beta \, \mu
&=
  \ln \rho - \sum_{n = 1}^\infty \beta_n \, \rho^n.
  \label{eq:series_murho}
\end{align}
are related by
\[
  (n + 1) \, B_{n+1} = -n \, \beta_n,
\]
for $n \ge 1$.


\problem{Pressure and chemical potential, II}



1. Similar to Eq. \eqref{eq:dPdrho}, show that
\begin{equation}
  \left(
    \frac{ \partial P } { \partial \mu }
  \right)_T
=
  \rho.
  \label{eq:dPdmu}
\end{equation}

\hint{
Start from $d(E - T \, S - \mu \, N) = -S \, dT - P \, dV - N \, \mu$.
}

2. Using Eq. \eqref{eq:dPdmu}, show that pressure assume the activity series
\begin{align}
  \beta \, P = \sum_{n = 1}^\infty b_n \, z^n,
  \label{eq:series_Pz}
\end{align}
then the activity series of density is given by
\begin{align}
  \rho = \sum_{n = 1}^\infty n \, b_n \, z^n.
  \label{eq:series_rhoz}
\end{align}
Equations \eqref{eq:series_Pz} and \eqref{eq:series_rhoz}
are the Mayer equations of state.

\hint{
  Use $z \propto \exp(\beta \, \mu)$
  and $\partial / \partial \mu = z \, \partial / \partial z$.
}



\problem{Density and activity expansions}



1. From Eqs. \eqref{eq:series_rhoz} and \eqref{eq:series_murho},
show that
\begin{align}
  n^2 \, b_n
=
  \frac{1} {(n - 1)!}
  \frac{ d^{n-1} } { d \rho^{n-1} }
  \,
  \exp
  \left(
  n \sum_{k = 1}^\infty \beta_k \, \rho^k
  \right).
\end{align}
In other words,
$n^2 \, b_n$
is the coefficient of $\rho^{n-1}$
of the density series of
$\exp \left( n \sum_{k = 1}^\infty \beta_k \, \rho^k \right)$.

\hint{
  From \eqref{eq:series_rhoz},
\begin{align*}
  n \, b_n
&= \oint
  \frac{ \rho } { z^{n+1} }
  \frac{ d z } { 2 \pi i }
= -\frac{1} {n}
  \oint
  \frac{ \rho }{ 2 \pi i }
  \, d \left( \frac{ 1 } { z^n } \right) \\
&=
  \frac{1} {n}
  \oint
  \frac{ 1 } { z^n } \, \frac{ d\rho }{ 2 \pi i },
\end{align*}
where the contour is around the origin of the complex plane.
%
Then use \eqref{eq:series_murho}:
$z = \rho \, \exp\left( \sum_{k=1}^\infty -\beta_k \, \rho^k \right)$.
}

2. Conversely, show that
\begin{align}
  -n \, \beta_n
=
  \frac{1} {n!}
  \frac{ d^{n} } { d z^{n} }
  \,
  \frac{1}
  {
  \bigl(
    \sum_{k = 1}^\infty k \, b_k \, z^{k-1}
  \bigr)^n
  },
\end{align}
or,
$-n \, \beta_n$
gives the coefficient of $z^n$
of the activity series of
$\left( \sum_{k = 1}^\infty k \, b_k \, z^{k-1} \right)^{-n}$.




\problem{Thermodynamic integration}



1. Show that the excess (non-ideal-gas) chemical potential $\muex$ can be written as
\begin{equation}
  \muex
=
  \int_0^1 d\xi \left\langle \frac{d U(\xi)}{d\xi} \right\rangle,
  \label{eq:muti}
\end{equation}
%
where the potential energy function is given by
\[
  U =
  \sum_{j = 2}^n u_\xi(r_{1j})
  +
  \sum_{i = 2}^{n - 1} \sum_{j = i+1}^n u(r_{ij}).
\]
%
Here, $\xi$ is called the ``charging'' parameter.
%
The charging parameter affects only the potential terms involving
the designated particle 1;
$\xi = 0$ means particle 1 is not interacting with other particles,
$\xi = 1$ means particle 1 is identical to any other particle.
%
That is,
$u_{\xi = 0}(r_{1j}) = 0$,
and
$u_{\xi = 1}(r_{1j}) = u(r_{1j})$.

2. Further, write the above expression as an integral of
the pair correlation function $g(\vr)$:
\begin{equation}
  \muex
=
  \rho \int_0^1 d\xi \int \frac{d u(\xi)}{d\xi} g_\xi(\vr) \, d\vr,
  \label{eq:mutigr}
\end{equation}



\problem{Free energy perturbation}

Show that the chemical potential can be written as
\begin{equation}
  \exp(-\beta \muex)
=
  \bigl\langle
    \exp( -\beta \Delta U )
  \bigr\rangle,
\end{equation}
%
where $\Delta U$ is the increment of the potential energy
after adding a particle into the system.



\chapter{Functional differentiation}


%\problem{Second derivative}
%
%
%Show that, in the grand-canonical ensemble,
%\begin{align}
%  & \frac{ \delta^2 \rho(1)}
%        { \beta^2 \delta \psi(2) \delta \psi(3)}
%           %\notag \\
%           = \rho^{(3)}(1, 2, 3)
%  \notag \\
%  &   - \rho^{(2)}(1, 2) \rho(3)
%     - \rho^{(2)}(2, 3) \rho(1)
%     - \rho^{(2)}(3, 1) \rho(2)
%     + 2 \rho(1) \rho(2) \rho(3) \notag \\
%   &
%     + \big[ \rho^{(2)}(1, 2) - \rho(1) \rho(2) \big] \delta(2, 3)
%     + \big[ \rho^{(2)}(2, 3) - \rho(2) \rho(3) \big] \delta(3, 1) \notag \\
%   &
%     + \big[ \rho^{(2)}(3, 1) - \rho(3) \rho(1) \big] \delta(1, 2) \notag \\
%   &
%     + \rho(1) \big[\delta(1, 2) \delta(2, 3)
%     + \delta [\delta(1,2)] / \delta [\delta(3)] \big].
%\end{align}
%%
%The notation here is the same as that of Ref. \cite{hansen} (Sec. 3.3, page 53).



\problem{Lovett-Mou-Buff formula}



Show, in an inhomogeneous liquid under the external field $\phi(\vr_1)$, that
%
\begin{align*}
  \nabla_1 \big[
    \rho(\vr_1)
  + \beta \, \rho(\vr_1) \, \phi(\vr_1)
\big]
  =
  -\int d \vr_2 \,
    \left[
      \rho^{(2)}(\vr_1, \vr_2) - \rho(\vr_1) \, \rho(\vr_2)
    \right]
    \nabla_2 \phi(\vr_2),
\end{align*}
%
and
%
\begin{align}
  \nabla_1 \big[
    \log \rho(\vr_1) + \beta \, \phi(\vr_1)
  \big]
   = \int d \vr_2 \,
    c(\vr_1, \vr_2)
    \nabla_2 \rho(\vr_2).
\end{align}
These are the Lovett-Mou-Buff(-Wertheim) equations\cite{lovett1976, wertheim1976}.


\hint{
Consider a perturbation $\rho(\vr_1 + \vct{d}) - \rho(\vr_1)$,
with $\vct{d}$ being a small displacement.
}


Further, rewrite the above equation as
%
\[
   \nabla_1
   \big[
     \log \rho(\vr_1) + \beta \, \phi(\vr_1)
   \big]
  =
 \int d \vr_2 \,
    \nabla_1 c(\vr_1, \vr_2)
    \rho h(\vr_2).
\]
%
Then, compare the above result with the Yvon-Born-Green equation:
%
\begin{align}
   \nabla_1
   \big[
     \log \rho(\vr_1) + \beta \, \phi(\vr_1)
   \big]
  =
 \int d \vr_2 \,
    \big[ \nabla_1 f(\vr_1, \vr_2) \big]
    \, y(\vr_1, \vr_2)
    h(\vr_2).
\end{align}




\problem{Electrostatic energy of a dipolar system}

The electrostatic energy can be computed from
\begin{equation}
  E = \frac{1}{2} \int \rho(\vr) \, \phi(\vr) \, d\vr,
  \label{eq:electrostaticenergy}
\end{equation}
where
\[
  \rho(\vr) = \int q_i \, \delta(\vr - \vr_i) \, d\vr.
\]

For a system of dipoles, we define the dipole density
\[
  \vct\upmu(\vr) = \int \vct\upmu_i \, \delta(\vr - \vr_i) \, d\vr,
\]
where $\vct\upmu_i = q_i \vct{d}_i$,
and $\vct{d}_i$ is the vector from the negative charge to the positive charge.
%
Then, show that \eqref{eq:electrostaticenergy} can be rewritten as
\[
  E = \frac{1}{2} \int \vct\upmu(\vr) \cdot \nabla \, \phi(\vr) \, d\vr.
\]

\hint{
  Try to write down the charge distribution of the dipole system.
}



\problem{Generating closures by the variational principle}
%Stell



\chapter{Integral equations}



\section{Closure}



\problem{Upper bound of the bridge function}

Show that the bridge function $B(\vr)$ satisfies\cite{kast2012}

\begin{equation}
  B(\vr) \le c(\vr) + \beta u(\vr).
\end{equation}

\hint{
  Use
\[
  g(\vr) = \exp[-\beta u(\vr) + t(\vr) + B(\vr)],
\]
and $1 + x \le \exp(x)$.}



\problem{Different closures, same result}

Albus exactly solved the $g(r)$ of liquid $A$ whose molecular potential is $u_A(r)$;
Harry solved that of liquid $H$ with $u_H(r)$ under the HNC closure;
Percy solved that of liquid $P$ with $u_P(r)$ under the PY closure.
Amazingly, they found that their results were the same.
Show that the bridge function of liquid $A$ satisfies
\begin{equation}
  -\beta u_A(r) + B_A(r) = -\beta u_H(r).
\end{equation}
Can you also deduce $u_P(r)$?



\section{Thermodynamic quantities}



\problem{Pressure in the PY closure}

In the PY closure, the pressure can be written as\cite{baxterpressure}
\begin{align}
  \beta P =  \int \left[
    \log(1-\rho \,\tilde{c}) + \frac{ \rho \, \tilde{c} }{2}
  \right] \, \dvk
  +
  \frac{\rho}{2}
  - \frac{\rho^2}{2} \int c(\vr) \, d\vr.
  \label{eq:Pbaxter}
\end{align}
%
By differentiating $\beta P$ with respect to $\rho$,
show that
\begin{align}
  \partial_\rho (\beta P) = 1 - \rho \int c(\vr) \, d\vr.
\end{align}

\hint{
  Define
\begin{align*}
  \tilde{T}
\equiv
  \log(1 - \rho \tilde{c})
  + \rho \tilde{c},
 % \label{eq:logrhoc}
\end{align*}
show that
%
\begin{align*}
  -\partial_\rho \tilde T
=
  \rho \, \tilde{h}
  \, \partial_\rho ( \rho \, \tilde{c} ),
\end{align*}
%
and
%
\begin{align}
  -\int \partial_\rho \tilde T \dvk
=
  \rho \int h \, c \, d\vr
  + \rho^2 \int h \, \partial_\rho c \, d\vr.
\label{eq:intdlogrhoc}
\end{align}
%
Also use the fact that $g(\vr) = h(\vr) + 1 = 0$ at $\vr = \vct 0$.
}



\problem{Chemical potential in the HNC closure}

Show that the excess chemical potential $\muex$ in the HNC closure
can be found from\cite{morita1960, singer1985}
\begin{equation}
  -\beta \muex = \rho \int \left( c - \tfrac{1}{2} h \, t\right) d\vr.
  \label{eq:muexhnc}
\end{equation}

\hint{
  Start from \eqref{eq:mutigr}.
}


\problem{Free energy in the HNC closure}

1. Show that the excess free energy in the HNC closure
can be found from\cite{morita1958, morita1960, singer1985}
\begin{align}
    - \frac{ \beta \Fex } { V }
    = \frac{\rho^2}{2} \int \left( c - \tfrac{1}{2} h^2 \right) \, d\vr
      - \frac{1}{2} \int  \big[
              \log\left( 1 - \rho \, \tilde{c} \right)
                           + \rho \, \tilde{c}
                         \big]  \dvk,
\label{eq:Fexhnc}
\end{align}

2. Differentiating \eqref{eq:Fexhnc} with respect to $\rho$,
show that it is identical to \eqref{eq:muexhnc}.

\hint{
  Use \eqref{eq:intdlogrhoc}.
}

%\begin{align*}
%  \partial_\xi g
%  &= (\partial_\xi f) \, y + (1+f) \, \partial_\xi y
%  = (\partial_\xi f) \, y + (1+f) \, y \, \partial_\xi t \\
%  &= (\partial_\xi f) \, y + g \, \partial_\xi t,
%\end{align*}
%Here, we have used $y = \exp t$ so that $\partial_\xi y = y \, \partial_\xi t$.
%
%Next, we will use
%\begin{align*}
%  -\beta \muex
%  = \rho \int_0^1 d\xi \int_\vr (\partial_\xi f) \, y \, d\vr \\
%\end{align*}
%where $\rho$ comes about such that
%\[
%  \rho \int_\vr f \, y \, d \vr = \rho \int_\vr g \, d\vr = N
%  \approx N - 1,
%\]
%which gives the number of particles interacting with particle 1.
%%
%Then,
%\begin{align*}
%  -\beta \muex
%  &= \rho \int_0^1 d\xi \int_\vr (\partial_\xi g - g \partial_\xi t) \, d\vr \\
%  &= \rho \int_0^1 d\xi \int_\vr (\partial_\xi c - h \partial_\xi t) \, d\vr \\
%  &= \rho \int_0^1 d\xi \int_\vr \partial_\xi \left(c - \tfrac{1}{2} h \, t \right) \, d\vr \\
%  &= \rho \int_\vr \left(c - \tfrac{1}{2} h \, t \right) \, d\vr
%\end{align*}
%We have assumed infinite dilution on the third line so that
%$t \, \partial_\xi h = h \, \partial_\xi t = \tfrac{1}{2} \partial_\xi (h t)$.
%
%



\bibliography{../liquid}
\bibliographystyle{alpha}
\end{document}

