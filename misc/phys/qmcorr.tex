\documentclass{article}
\usepackage{amsmath}
\usepackage{bm}
\usepackage[usenames,dvipsnames]{xcolor}
\usepackage{tikz}
\usepackage{hyperref}

\hypersetup{
    colorlinks,
    linkcolor={red!30!black},
    citecolor={blue!50!black},
    urlcolor={blue!80!black}
}



\begin{document}

\title{Quantum mechanical correction and MD time step}
\author{ \vspace{-10ex} }
\date{ \vspace{-10ex} }
\maketitle


\section{Quantum mechanical correction}


The aim is to obtain first quantum mechanical corrections
to a classical system.
%
We only summarize the result.
%
The detailed derivation can be found in
Statistical Physics by Landau and Lifshits \S 33 (pages 99-102).


The partition function can be written as
\begin{equation}
Z = \int
\exp\left[-\beta H(q, p) + \hbar^2 \chi_2 \right] \, dq \, dp.
\end{equation}
where
\begin{align}
\chi_2
&=
-\frac{ \beta^4 } { 8 }
\left(
  \sum_i \frac{p_i}{m_i} \frac{\partial U}{\partial q_i}
\right)^2
+\frac{ \beta^3 } { 6 }
\sum_{i,k} \frac{p_i}{m_i} \frac{p_k}{m_k}
  \frac{\partial^2 U}{\partial q_i \partial q_k}
  \notag \\
&\quad
+\frac{ \beta^3 }{6}
  \sum_i \frac{1}{m_i}
    \left( \frac{ \partial U } { \partial q_i } \right)^2
-\frac{ \beta^2 }{4}
  \sum_i \frac{1}{m_i}
    \frac{ \partial^2 U } { \partial q_i^2 }.
\end{align}
Thus, this expression gives a modification to Boltzmann weight.

If we average out the momenta components, then
\begin{align}
\chi_2(q)
&=
-\frac{ \beta^3 } { 8 }
  \sum_i \frac{1}{m_i} \left( \frac{\partial U}{\partial q_i} \right)^2
+\frac{ \beta^2 } { 6 }
\sum_{i} \frac{1}{m_i}
  \frac{\partial^2 U}{\partial q_i^2}
  \notag \\
&\quad
+\frac{ \beta^3 }{6}
  \sum_i \frac{1}{m_i}
    \left( \frac{ \partial U } { \partial q_i } \right)^2
-\frac{ \beta^2 }{4}
  \sum_i \frac{1}{m_i}
    \frac{ \partial^2 U } { \partial q_i^2 }
\notag \\
&=
\frac{ \beta^3 }{24}
  \sum_i \frac{1}{m_i}
    \left( \frac{ \partial U } { \partial q_i } \right)^2
-\frac{ \beta^2 }{12}
  \sum_i \frac{1}{m_i}
    \frac{ \partial^2 U } { \partial q_i^2 }.
\end{align}
This means the effective potential is
%
\begin{equation}
U_{\mathrm{eff}}(q)
=
U(q)
+
\frac{ \beta^2 \, \hbar^2 }{24}
\sum_i \frac{1}{m_i}
\left[
    \frac{2}{\beta}
    \frac{ \partial^2 U } { \partial q_i^2 }
-
    \left( \frac{ \partial U } { \partial q_i } \right)^2
\right].
\label{eq:Ueff}
\end{equation}


Similarly,
if we average over coordinates,
\begin{align}
\chi_2(p)
&=
-\frac{ \beta^4 } { 8 }
\sum_{i,k} \frac{p_i}{m_i} \frac{p_k}{m_k}
  \left\langle
    \frac{\partial U}{\partial q_i}
    \frac{\partial U}{\partial q_k}
  \right\rangle
+\frac{ \beta^3 } { 6 }
\sum_{i,k} \frac{p_i}{m_i} \frac{p_k}{m_k}
  \left\langle
    \frac{\partial^2 U}{\partial q_i \partial q_k}
  \right\rangle
+ C
\notag \\
&=
\frac{ \beta^4 } { 12 }
\sum_{i,k} \frac{p_i}{m_i} \frac{p_k}{m_k}
  \left\langle
    \frac{\partial U}{\partial q_i}
    \frac{\partial U}{\partial q_k}
  \right\rangle
+ C,
\end{align}
where we have used
$$
\int
\frac{ \partial^2 U } { \partial q_i \partial q_k} e^{-\beta U} \, dq
=
-\int
\frac{ \partial U } { \partial q_k}
\frac{ \partial e^{-\beta U} } { \partial q_i } \, dq
=
\beta \int
\frac{ \partial U } { \partial q_k}
\frac{ \partial U } { \partial q_i}
e^{-\beta U} \, dq,
$$
or
$$
\left\langle
  \frac{ \partial^2 U } { \partial q_i \partial q_k}
\right\rangle
=
\beta \,
\left\langle
  \frac{ \partial U } { \partial q_k}
  \frac{ \partial U } { \partial q_i}
\right\rangle.
$$


\section{MD time step}


In molecular dynamics, Newton's equations are discretized
by integrators.
For the velocity Verlet integrator
$$
\begin{aligned}
p_i &\rightarrow  p_i - \frac{ \partial U } { \partial q_i } \frac{\Delta t}{2} \\
q_i &\rightarrow  q_i + \frac{ p_i } { m_i } \Delta t \\
p_i &\rightarrow  p_i - \frac{ \partial U } { \partial q_i } \frac{\Delta t}{2},
\end{aligned}
$$
One can show that this dynamics preserves an approximate Hamiltonian,
called shadow (perturbed, modified) Hamiltonain,
which is given by
$$
\tilde H(q, p)
=
H(q, p)
+
\frac{ \Delta t^2 }{12}
\sum_{ik} \frac{p_i}{m_i}
    \frac{p_k}{m_k}
    \frac{ \partial^2 U } { \partial q_i \partial q_k }
-
\frac{ \Delta t^2 } { 24 }
\sum_i \frac{1}{m_i}
    \left( \frac{ \partial U } { \partial q_i } \right)^2.
$$
If we average out the momenta
with $\langle p_i \, p_k \rangle = m_i \, \delta_{ik} / \beta$,
then
$$
\tilde U(q)
=
U(q)
+
\frac{ \Delta t^2 }{ 24 }
\sum_{i}
  \frac{1}{m_i}
  \left[
  \frac{2}{\beta}
    \frac{ \partial^2 U } { \partial q_i^2 }
-
    \left( \frac{ \partial U } { \partial q_i } \right)^2
\right].
$$
This is very similar to the effective potential from the quantum correction.
%
This means if choose $\Delta t = \beta \hbar$,
then we can successfully take into account the quantum mechanical effect.



Let us estimate this lucky $\Delta t$.
%
With
the reduced Planck constant
$\hbar = 1.054571800\times10^{-34} \mathrm{J}\cdot \mathrm{s}$,
the Boltzmann constant
$k_B = 1.38064852 \times 10^{-23}\mathrm{J}/\mathrm{K}$,
we find that at $T = 300 \mathrm{K}$,
$
\Delta t^* = 25.46 \times 10^{-15}\mathrm{s} \approx 25 \, \mathrm{fs}.
$
This is unfortunately 12 times as larger as the usual
MD time step $2 \, \mathrm{fs}$.


\end{document}
