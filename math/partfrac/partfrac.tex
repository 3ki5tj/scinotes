\documentclass{article}
\usepackage{amsmath}
\begin{document}

The problem is to express $\frac{x - 1}{(x+3)(x^2+1)}$
into partial fractions as
\begin{equation}
\frac{x - 1}{(x+3)(x^2+1)} =
\frac{A}{x+3} + \frac{Bx+ C}{x^2+1}.
\label{eq:partfrac}
\end{equation}

Here is my favorite way of resolving $A$, $B$ and $C$.
We start with Eq. \eqref{eq:partfrac}.
Multiplying both sides by $x+3$ we get
$$
\frac{x - 1}{(x^2+1)} =
A + \frac{Bx+ C}{x^2+1} (x+3).
$$
Now taking the limit of $x \rightarrow -3$ yields
\begin{equation}
A = \frac{-3-1}{(-3)^2 + 1} = -\frac{2}{5}.
\end{equation}

Similarly, multiplying both sides by $x^2 + 1$ yields
$$
\frac{x - 1}{(x+3)} =
\frac{A}{x+3}(x^2+1) + Bx+ C.
$$
Taking the limit of $x\rightarrow i$ yields
$$
B i + C = \frac{i - 1}{i + 3} = -\frac{1}{5} + \frac{2 i}{5}.
$$
Taking the complex conjugate (or doing the same thing for $x \rightarrow -i$) yields
$$
-B i + C = \frac{i - 1}{i + 3} = -\frac{1}{5} - \frac{2 i}{5}.
$$
So
\begin{align}
B &= \frac{2}{5} \\
C &= -\frac{1}{5}
\end{align}

\end{document}
