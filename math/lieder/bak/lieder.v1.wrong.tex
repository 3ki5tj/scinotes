\documentclass{article}
\usepackage{amsmath}
\usepackage{bm}
\usepackage[usenames,dvipsnames]{xcolor}
\usepackage{tikz}
\usepackage{hyperref}

\hypersetup{
    colorlinks,
    linkcolor={red!30!black},
    citecolor={blue!50!black},
    urlcolor={blue!80!black}
}


\begin{document}


\title{A primer on the Lie derivative}
\author{ \vspace{-10ex} }
\date{ \vspace{-10ex} }
\maketitle


The purpose of this article is to give a
gentle introduction to the idea of
Lie derivative and operators.
%
Basically, a Lie derivative is an operator,
e.g., a partial derivative $\partial/\partial x$,
or something fancier like $y \, \partial/\partial x$.
%
When this operator is applied to a function,
it generates an another function.
%
We can also build more complex differential operators, like
$\exp(s \, x \partial/\partial x)$.
%
As we shall see,
these operators allows us to
geometrically manipulate the shape
of a function through rotation, twisting, etc.

In application,
differential operators are commonly used classical mechanics
and statistical mechanics.
%
A prominent example is the Liouville operator
which represents the time evolution of
a quantity (an observable or a distribution).

Below, we shall introduce these ideas through
a simple problem of manipulating the shape of a function.



\section{Reshaping a function a through operators}


A function
\begin{equation}
  y = f(x)
  \label{eq:y_fx}
\end{equation}
can be regarded as a curve of $y$ versus $x$.
Let us consider the problem of changing the shape of this curve.


\subsection{Operations along the $y$ axis}

To translate $f(x)$ along the $-y$ axis by $b$,
we only need to add a constant $c$ to it:
\begin{equation}
  f(x) \rightarrow f(x) + c.
  \label{eq:ys_shift}
\end{equation}
So we can write this operation through an operator
\begin{equation}
  T_y^* f \equiv f + c.
  \label{eq:ys_shift2}
\end{equation}

To scale it along the $y$ axis, we only need to multiply it
by a constant $A$, that is
\begin{equation}
  f(x) \rightarrow A \, f(x).
  \label{eq:ys_scale}
\end{equation}
This can be written in terms of an operator
\begin{equation}
  S_y^* f \equiv A f.
  \label{eq:ys_scale2}
\end{equation}


\subsection{Operations along the $x$ axis}

But how about the same operations along the $x$ axis?

If we shift a function $f(x)$ to the \emph{left} by $a$,
the resulting function is $f(x + a)$.
Can we achieve this result by an operator?
The answer is yes!
Consider the Taylor expansion:
$$
f(x + a) = \sum_{n = 0}^\infty \frac{a^n}{n!} \frac{d^n f(x) }{dx^n}
\equiv \exp\left(a \frac{d}{dx} \right) f(x).
$$
The last step is a shortcut of
writing all orders of derivatives,
$$
T_x \equiv
\exp\left(a \frac{d}{dx} \right)
\equiv
\sum_{n = 0}^\infty \frac{a^n}{n!} \frac{d^n }{ dx^n}.
$$
We shall write convenient shortcut
a differential operator, $T_x$.
%
This operator obviously behaves kind of different from $T_y^*$,
so we shall drop the $*$ on the superscript to distinguish
the two.
%
Another point to note is that the $+$ sign in $T_y^*$
means moving the curve along $+y$ axis,
while the $+$ sign in $T_x$ means
moving the curve along $-x$ axis.
%
The point is that
applying $T_x$ to any function $f(x)$, we get $f(x+a)$.

Similarly, if we \emph{compress} the function along the $x$ axis by a factor of $s$,
then the resulting function is given by $f(s \, x)$.
To find the corresponding operator note that
$$
\begin{aligned}
f(s \, x)
&= f( e^{\ln x + \ln s} ) \\
&= \sum_{n = 0}^\infty \frac{(\ln s)^n}{n!} \frac{d^n f(e^{\ln x + \ln s}) }{ d(\ln x)^n } \\
&= \exp\left[ \ln s \frac{d}{d\ln x} f(x) \right]
= s^{x \frac{d}{dx} } f(x).
\end{aligned}
$$
So the scaling operation along the $x$ axis is given by
$$
S_x \equiv s^{x \frac{d}{dx} }.
$$



\subsection{Composition}


We can join two shaping operations by composition.
%Intuitively, this means two successive substitutions.


\subsubsection{Composition of operations along $y$}

For example, if we want to first translate the function by $c$ vertically,
and then stretch it by a factor of $A$ along the $y$ axis.
Then the resulting function is $A \, [f(x) + c]$
and it can be achieved by a composition of the operators
of $S_y^*$ and $T_y^*$:
\begin{equation}
S_y^* T_y^* f(x) = S_y^* [f(x) + c] = A \, [f(x) + c].
\label{eq:SyTy}
\end{equation}
Note that the first operation $T_y^*$ is placed on the right,
so it is closer to the function $f$.


\subsubsection{Composition of operations along $x$}

Now consider the corresponding composite operation along the $x$ axis.
%
If we translate the function to the left by $a$ (i.e.,
the argument of the function should be $x_1 \equiv x + a$)
and then compress the resulting curve along $x$ by a factor $s$ (i.e.,
the argument of the function should be $x_2 \equiv s x_1$),
the resulting function should be
$$
  f(s(x+a)).
$$
%
It is naturally to draw analogy from Eq. \eqref{eq:SyTy} that
the above expression can be obtained from
$$
S_x \, T_x \, f(x).
$$
But this is wrong, because
$$
S_x T_x f(x) = S_x f(x + a) = f(s x + a) \ne f(s (x + a)).
$$
The correct answer is
$$
T_x S_x f(x) = T_x f(s \, x) = f(s (x + a)).
$$

Admittedly, the reversal of ordering is generally true,
although it is somewhat surprising.
To understand it,
recall that the operators $T_x$ and $S_x$
are defined in terms of $x$,
so they always try to manipulate $x$
instead of the argument of the function.
Thus the first substitution $x_1 = x + a$
corresponds to the last (leftmost) differential operator
to the function such that this differential operator
can go into the deepest level of the argument.

Generally, if the argument is subject to a sequence of substitutions.
$$
\begin{aligned}
S_1: & x_1 = X_1(x) \\
S_2: & x_2 = X_2(x_1) \\
\dots \\
S_n: & x_n = X_n(x_{n-1})
\end{aligned}
$$
Then, the function $f(X_n(\cdots X_1(x)\cdots))$
can be obtained by
$S_1 S_2 \cdots S_n f(x)$.

Proof. It is clear that
$$
\begin{aligned}
S_n \, f(x) &= f(X_n(x))
\\
S_{n-1} \, S_n \, f(x) &= S_{n-1} f(X_n(x)) = f(X_n(X_{n-1}(x)))
\\
S_1 S_2 \cdots S_{n-1} \, S_n f(x) &= f(X_n(X_{n-1}(\cdots X_2(X_1(x))\cdots))).
\end{aligned}
$$


\subsubsection{Composition of operations along both $x$ and $y$}

Now how about a sequence of reshaping operations that involve both $x$ and $y$ operations?
%
For example,
what if we want to translate $f$ to the left by $a$,
then scale it along the $y$ axis by $A$,
then compress it along the $x$ axis by a factor of $s$,
and finally translate it upward by $c$.

To solve such a problem, we first write down the desired function
$$
\begin{aligned}
\mathrm{step}\; 1:\;& f(x+a) \\
\mathrm{step}\; 2:\;& A \, f(x+a) \\
\mathrm{step}\; 3:\;& A \, f(s(x+a)) \\
\mathrm{step}\; 4:\;& A \, f(s(x+a)) + c
\end{aligned}
$$
It is readily seen that this result
can be obtained by $T_y^* T_x S_x S_y^*$,
$$
\begin{aligned}
T_y^* T_x S_x S_y^* f(x)
&= T_y^* T_x S_x A \, f(x) = T_y^* T_x A \, f(s x) \\
&= T_y^* A \, f(s(x+a)) = A \, f(s(x+a)) + c.
\end{aligned}
$$
But the same result can also be obtained by
$T_y^* S_y^* T_x S_x$ or $T_x S_x T_y^* S_y^*$.
$$
\begin{aligned}
T_y^* S_y^* T_x S_x f(x)
&= T_y^* S_y^* T_x f(s \, x) = T_y^* S_y^* f(s (x + a) ) \\
&= T_y^* A \, f(s(x+a)) = A \, f(s(x+a)) + c.
\end{aligned}
$$
or
$$
\begin{aligned}
T_x S_x T_y^* S_y^* f(x)
&= T_x S_x T_y^* A \, f(x) = T_x S_x [A \, f(x) + c] \\
&= T_x [A \, f(s \, x) + c] = A \, f(s(x+a)) + c.
\end{aligned}
$$

This illustrates a general result:
since the $x$ and $y$ operators commute,
we can exchange an $x$ operator
and an $y$ operator without affecting result.
%
We only need to ensure that the $y$ operators are arranged from right to left,
and the $x$ operators from left to right.


\subsection{Unification of $x$ and $y$ operations}

The asymmetry between $x$ and $y$ operations
is somewhat frustrating.
%
Besides, $x$ and $y$
operations appear to have different signs:
moving in the $+y$ direction was associated with $+c$,
whereas moving in the $-x$ direction was associated with $+a$.
So is there a way to unify the two?
The answer is again yes.
The reason of asymmetry is that
we write the curve in terms of $y$ as a function of $x$.
%
If we express $x$ and $y$ in equal footing,
then the asymmetry disappears.

Consider the writing the function
$$
F(x, y) = f(x) - y = 0.
$$
This is a two-variable function
in which $x$ and $y$ are symmetric,
although it means pretty much the same thing as
Eq. \eqref{eq:y_fx}.
So we may guess that in this form
the $y$ operations are the same as the $x$ operations.
Let us verify that.

First, after a shift along the $-y$ axis by $a_y$, we expect
the function $F(x, y)$ to become $F(x, y + a_y)$
$$
F(x, y + a_y) = f(x) - (y + a_y) = 0.
$$
This means
$$
  y = f(x) - a_y.
$$
which agrees with Eq. \eqref{eq:ys_shift} with $c = - a_y$
(i.e., moving the curve along $+y$ by $c = -a_y$).
But now this operation can be accomplished with an $x$-type
operator applied to $F(x, y)$,
\begin{equation}
  T_y \equiv
  \exp\left(a_y \frac{ \partial } { \partial y } \right).
  \label{eq:y_shift}
\end{equation}
We can readily verify
$$
\begin{aligned}
T_y F(x, y)
&= \exp\left(a_y \frac{ \partial } { \partial y } \right) F(x, y) \\
&= \sum_{n = 0}^\infty \frac{ a_y^n }{n!}
\frac{ \partial^n } {\partial y^n} F(x, y) \\
&= F(x, y + a_y).
\end{aligned}
$$


Second,
a compression of the curve
along the $y$ axis by a factor of $s_y$,
can be accomplished by
\begin{equation}
  S_y \equiv
  (s_y)^{ y \, \frac{ \partial } { \partial y } }.
  \label{eq:y_scale}
\end{equation}
Then
$$
S_y \, F(x, y)
=
F(x, s_y y)
= f(x) - s_y \, y.
$$
This is equivalent to
$$
y = \frac{1}{s_y} f(x),
$$
which agrees with Eq. \eqref{eq:y_shift} with $A = 1/s_y$.


Lastly, the ordering of operators now follows
the $x$ form as well.
Consider the composition of a $y$ translation
followed by a $y$ scaling.
This should be accomplished by applying $T_y \, S_y$ on $F(x, y)$
(cf. Eq. \eqref{eq:SyTy} for $S_y^* T_y^* f$):
$$
T_y \, S_y \, F(x, y)
= T_y F(x, s_y \, y)
= F(x, s_y (y + a_y))
= f(x) - s_y (y + a_y).
$$


\end{document}
