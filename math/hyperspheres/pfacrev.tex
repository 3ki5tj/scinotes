\documentclass{article}
\usepackage{amsmath}
\begin{document}

In this paper, the authors proposed a simple and clever method of predicting
the random close packing density using the Percus-Yevick (PY) integral equation
theory.
The idea is that under a sufficiently high density, the PY theory breaks down
and produces negative values of radial distribution function, $g(r)$, even
though the equation of state has yet manifested any singularity
at such a density.
This is an astute and interesting observation by itself.
Then the authors argue that the above critical density can be identified as
the density of random close packing.
While the paper certainly presents a very thought-provoking idea in a clear
and elegant style, there are several points that may need further clarification.

First, the key observation of the paper is the fact that the PY theory can
produce negative values for $g(r)$.  But this is because the PY theory adopts
a linearized closure
\begin{equation}
g(r) = e^{-\beta u(r)} [ 1 + t(r) ],
\label{eq:grPY}
\end{equation}
out of, arguably, mathematical simplicity.
Here $\beta = 1/(kT)$ is the inverse temperature,
$u(r)$ and $t(r)$ are the pair potential and the indirect correlation
function, respectively.
Although the PY theory works remarkably well for the three-dimensional
hard-sphere fluid, one may argue that generally, Eq. \eqref{eq:grPY},
is only an expedient approximation to circumvent the problem of missing
the exact bridge function, $B(r)$.
In other words, negative values of $g(r)$ may show up because of an artifact
of the PY theory instead of physical reasons.
While the authors have given on the second page a nontraditional derivation
of the PY theory and some possible extensions, there appears to be no direct
explanation for the physical nature of negative values of $g(r)$.
Thus, it is somewhat difficult to estimate the applicability of the proposed
method to other fluids.


Second, the extrapolation method employed by the authors in computing $g(r)$
for even dimensions might be a bit dangerous.
This is especially so for higher dimensions because the radius of convergence
of the density series can be much smaller than the random close-packing
density reported in Table I of the paper.
The two densities can differ by several orders of magnitude with the ratio
shrinking exponentially with the dimensionality, $d$.
A safer alternative is to use an advanced iterative integral equation
solver, such as the MDIIS method\cite{pulay1980, pulay1982, kovalenko1999},
to directly compute $g(r)$.
This would avoid the extrapolation error.
Using this implementation, one can verify the figures reported in Table I for
odd dimensions.
For even dimensions, the Fourier transform can be implemented with the
discrete Hankel transform\cite{lado1967, johnson1987}.
The reported values for the packing fractions of $d = 6$ and $8$ appear to
be too small.
The new corrected figures (computed using the MDIIS method), attached in the
following table, appear to interpolate better with the values for odd
dimensions.

Third, the PY theory is known to be less reliable for higher dimensions.
The reason is that the missing bridge diagrams from Eq. \eqref{eq:grPY}
grow rapidly with the dimensionality, $d$.
In fact, the predicted $\eta_c$ appear to exceed the close packing fraction
for $d \ge 7$.
[If $\eta_c(d = 6)$ is corrected, it also exceeds the close packing fraction
in that dimension.]
While this fact does not affect the validity in three dimensions,
higher-dimensional cases may deserve some further remedies.

\begin{table*}[h]
\centering
\renewcommand{\arraystretch}{2.5}
\begin{tabular}{llll}
Dimension $d$     &   $\eta_\mathrm{CP}(d)$  & $\eta_c(d)$ \\
\hline
$2$     &   $\dfrac{ \pi } { 2 \, \sqrt 3 } \approx 0.907$      & ($0.785$) \\
$3$     &   $\dfrac{ \pi } { 3 \, \sqrt 2 } \approx 0.740$      & $0.613$ \\
$4$     &   $\dfrac{ \pi^2 } { 8 \, \sqrt{5} } \approx 0.552$   & $0.467$ ($0.462$) \\
$5$     &   $\dfrac{ \pi^2 } { 15 \, \sqrt 3 } \approx 0.380$   & $0.367$ \\
$6$     &   $\dfrac{ \pi^3 } { 48 \sqrt{7} } \approx 0.244$     & $0.230$ ($0.262$) \\
$7$     &   $\dfrac{ \pi^3 } { 210 } \approx 0.148$             & $0.207$ \\
$8$     &   $\dfrac{ \pi^4 } { 1152 } \approx 0.085$            & $0.087$ ($0.132$) \\
$9$     &   $\dfrac{ \pi^4 } { 945 \, \sqrt 5 } \approx 0.046$  & $0.112$
\end{tabular}
\caption{Close packing fractions, $\eta_\mathrm{CP}(d)$,
and the random close packing fractions, $\eta_c(d)$
for the hard-sphere fluid.
The latter were copied from Table I of the paper.
The corrected values, if any, are shown in parentheses. }
\end{table*}

\bibliographystyle{unsrt}
\bibliography{liquid}
\end{document}
