\documentclass{article}
\usepackage{amsmath}
\usepackage{amsthm}
\usepackage{amsfonts}
\usepackage{mathrsfs}
\usepackage{bm}
\usepackage{listings}
\usepackage[usenames,dvipsnames]{xcolor}
\usepackage{tikz}
\usepackage{hyperref}

\hypersetup{
    colorlinks,
    linkcolor={red!30!black},
    citecolor={blue!50!black},
    urlcolor={blue!80!black}
}


\newcommand{\vct}[1]{\mathbf{#1}}
\newcommand{\vx}{\vct{x}}
\newcommand{\vy}{\vct{y}}
\newcommand{\Z}{\mathcal{Z}}
\newcommand{\E}{\mathcal{E}}
\newcommand{\Ham}{\mathcal{H}}
\newcommand{\W}{\mathcal{W}}
\newcommand{\A}{\mathcal{A}}
\newcommand{\LL}{\mathcal{L}}
\newcommand{\var}{\mathrm{var}}
\newcommand{\com}{\mathrm{com}}

\newcommand{\llbra}{[\![}
\newcommand{\llket}{]\!]}

% annotation macros
\newcommand{\repl}[2]{{\color{gray} [#1] }{\color{blue} #2}}
\newcommand{\add}[1]{{\color{blue} #1}}
\newcommand{\del}[1]{{\color{gray} [#1]}}
\newcommand{\note}[1]{{\color{OliveGreen}\small [\textbf{Comment.} #1]}}


\definecolor{dkgreen}{rgb}{0,0.6,0}
\definecolor{gray}{rgb}{0.5,0.5,0.5}
\definecolor{mauve}{rgb}{0.58,0,0.82}

\lstset{frame=tb,
  language=C++,
  aboveskip=3mm,
  belowskip=3mm,
  showstringspaces=false,
  columns=flexible,
  basicstyle={\small\ttfamily},
  numbers=none,
  numberstyle=\tiny\color{gray},
  keywordstyle=\color{blue},
  commentstyle=\color{dkgreen},
  stringstyle=\color{mauve},
  breaklines=true,
  breakatwhitespace=true,
  tabsize=2
}


\begin{document}



\title{NAMD Hacking Notes}
\author{ \vspace{-10ex} }
\date{ \vspace{-10ex} }


\maketitle

\tableofcontents


\section{SimParameters}

Let us start with something easy--the input parameters.
%
The header file `SimParameters.h' effectively
lists every parameter we can use in the `.conf' configuration file.
%
For example,
\begin{lstlisting}
  Bool commOnly; // Don't do any force evaluations
  Bool tclForcesOn; // Are Tcl forces on?
  Bool colvarsOn;
\end{lstlisting}
%
These parameters are organized in a class,
and the instance \texttt{simParams}
(which is a nick name \texttt{Node::Object()->simParameters})
is available in both master and slave nodes (see later),
which means that the input files are accessible
by every computing node.


\section{Master/slave architecture}

NAMD is a parallel code of molecular dynamics (MD).
%
Different computer nodes do different jobs.

Tasks that can be parallelized are done on the slave nodes.
%
For example, integrating Newton's equation, computing the force and energy, etc.

Tasks that are not parallelized are done on the master node.
%
For example, reading the user input file,
writing the restart file, etc.

When the code is running, there are many slave nodes,
but only one master node.
%
In NAMD, the notion `node' is a virtual node,
and it does not match the number of CPU cores.
%
For example, you can run NAMD on a 16-core machine
for a large protein/water system;
NAMD may decide to use thousands of virtual slave nodes
instead of 15 or 16 of them.


\subsection{Master-slave communication}

Because of the master/slave architecture,
we need some tools to communicate between master and slave nodes.
%
Section \ref{sec:broadcasting} shows how the master node
can send information to the slave nodes (broadcasting).
%
Section \ref{sec:collection} shows how the slave nodes
can send information to the master.
%
Section \ref{sec:reduction} shows how the master node
can get the sum of an additive quantity
computed at slave nodes (called reduction).


\section{Controller (master) and Sequencer (slave)}

`Controller.C' is the code running on the master node.
`Sequencer.C' is the code running on the slave nodes.
For functions that require communication between
the master and slave nodes, e.g., \texttt{integrate()},
there is a version for the master node,
and another version for the slave nodes.

\subsection{Controller}

The most important function is \texttt{integrate()},
which runs the main MD loop,
this function is independent of the function of the same name
in \texttt{Sequencer.C},
which is for the slave nodes.

\begin{lstlisting}
void Controller::integrate(int scriptTask) {
  for ( ++step ; step <= numberOfSteps; step++ ) {
    adaptTempUpdate(step);
    tcoupleVelocities(step);
    berendsenPressure(step);
    enqueueCollections(step);  // collect positions, velocities, forces
    receivePressure(step);
    if ( zeroMomentum && dofull && ! (step % slowFreq) )
      correctMomentum(step);
    ...
#if STEP_BARRIER
    cycleBarrier(1, step);
#endif
    rebalanceLoad(step);
  }
}
\end{lstlisting}
%
Now notice that the master node does not actually
integrate Newton's equation of motion,
because it is a job of slaves.
%
All the master does is to collect data from slaves
and assist some tricky steps for the thermostat and barostat.
%
It also does the writing of restart files.
%
For example, for adaptive tempering,
it is done in \texttt{adaptTempUpdate(step)}.

The `Controller' has a few handy variables,
they are quite self explanatory.
\begin{lstlisting}
  NamdState *const state;
  BigReal totalEnergy;
  BigReal kineticEnergy;
  int     numDegFreedom;
  Tensor  pressure;
\end{lstlisting}
%
For example, \texttt{state->pdb} contains information
of the initial PDB, which can help determine the number of atoms
via \texttt{state->pdb->num\_atoms()}, or the position of the \texttt{i}th atom
via \texttt{state->pdb->get\_position\_for\_atom(pos, i)}.
Similarly, \texttt{state->molecule} contains information of charge and mass,
e.g., \texttt{state->molecule->atommass(i)} gives the mass
of the \texttt{i}th atom.

The data collection part is done by the function
\texttt{enqueueCollections()}.

\begin{lstlisting}
// collect positions, velocities, forces from slave nodes
void Controller::enqueueCollections(int timestep)
{
  if ( Output::coordinateNeeded(timestep) ){
    collection->enqueuePositions(timestep,state->lattice);
  }
  if ( Output::velocityNeeded(timestep) )
    collection->enqueueVelocities(timestep);
  if ( Output::forceNeeded(timestep) )
    collection->enqueueForces(timestep);
}
\end{lstlisting}


\subsection{Sequencer}

The slave version of function \texttt{integrate()}
is much more complex.  Below is a simplified version.

\begin{lstlisting}
void Sequencer::integrate(int scriptTask) {
  for ( ++step; step <= numberOfSteps; step++ ) {
    rescaleVelocities(step);
    tcoupleVelocities(timestep,step);
    berendsenPressure(step);

    if ( ! commOnly )
      addForceToMomentum(0.5*timestep);
    ...
    if ( ! commOnly ) addVelocityToPosition(0.5*timestep);
    ...
    langevinPiston(step);
    if ( ! commOnly ) addVelocityToPosition(0.5*timestep);
    ...
    submitHalfstep(step);
    ...
    runComputeObjects(!(step%stepsPerCycle),step<numberOfSteps);

    // reassignment based on full-step velocities
    if ( !commOnly && ( reassignFreq>0 ) && ! (step%reassignFreq) ) {
      reassignVelocities(timestep,step);
      addForceToMomentum(-0.5*timestep);
      rattle1(-timestep,0);
    }

    if ( ! commOnly ) {
      langevinVelocitiesBBK1(timestep);
      addForceToMomentum(timestep);
      langevinVelocitiesBBK2(timestep);
    }
    ...
    rattle1(timestep,1);
    ...
    submitHalfstep(step);
    if ( zeroMomentum && doFullElectrostatics ) submitMomentum(step);
    ...
    if ( ! commOnly ) {
      addForceToMomentum(-0.5*timestep);
    }
    ...
    submitReductions(step);
    submitCollections(step);
    adaptTempUpdate(step);
    ...
#if  STEP_BARRIER
    cycleBarrier(1, step);
#endif
    rebalanceLoad(step);
  }
}
\end{lstlisting}




\section{\label{sec:collection}CollectionMaster (master) and CollectionMgr (slave)}


We will focus on the non-\texttt{MEM\_OPT\_VERSION}.

\subsection{CollectionMaster}

The most important function is
\texttt{CollectionMaster::enqueuePositions()},
which is used in \texttt{Controller.C}.

\begin{lstlisting}
void CollectionMaster::receivePositions(CollectVectorMsg *msg)
{
  positions.submitData(msg,Node::Object()->molecule->numAtoms);
  delete msg;

  CollectVectorInstance *c;
  while ( ( c = positions.removeReady() ) ) { disposePositions(c); }
}

// request from slaves the coordinates in step `seq'
// use the coordinates (for file writing, TCL scripting, etc),
// and dispose of it
void CollectionMaster::enqueuePositions(int seq, Lattice &lattice)
{
  positions.enqueue(seq,lattice);

  CollectVectorInstance *c;
  while ( ( c = positions.removeReady() ) ) { disposePositions(c); }
}

// write and use the position saved in the
// CollectVectorInstance *c, and then delete it
void CollectionMaster::disposePositions(CollectVectorInstance *c)
{
  int seq = c->seq;
  int size = c->data.size();
  if ( ! size ) size = c->fdata.size();
  Vector *data = c->data.begin();
  FloatVector *fdata = c->fdata.begin();
  Node::Object()->output->coordinate(seq,size,data,fdata,c->lattice);
  c->free();
}
\end{lstlisting}




\section{Output.C}

The \texttt{output->coordinate()} function is used to write coordinates.
%
This function is used in \texttt{CollectionMaster::disposePosition()}
defined in \texttt{CollectionMaster.C}.

\begin{lstlisting}
void Output::coordinate(timestep, n, Vector coor, FloatVector fcoor,
                        Lattice &lattice)
{
  // Output to DCD trajectory use the float coordinates
  if ( timestep % simParam->dcdFrequency == 0 ) {
    wrap_coor(fcoor, lattice, ...);
    output_dcdfile(timestep, n, fcoor, &lattice);
  }

  // Output a restart file
  if ( timestep % simParams->restartFrequency == 0 ) {
    wrap_coor(coor, lattice, ...);
    output_restart_coordinates(coor, n, timestep);
  }
  ...

  // TCL scripts
  if ( timestep == EVAL_MEASURE ) {
    wrap_coor(coor, lattice, ...);
    Node::Object()->getScript()->measure(coor);
  }
  ...
}
\end{lstlisting}


\section{HomePatch}

This is a patch of atoms that are handled by a computing node.
%\bibliographystyle{plain}
%\bibliography{simul}


\section{\label{sec:reduction}Summing parts, ReductionMgr}

In most cases,
``reduction'' means adding things up.
%
For example, each slave node has computed
a part of the potential energy,
the total potential energy is the sum of them.
%
Then after a reduction,
the \emph{master} node gets the total potential.

A limitation is that the reduced thing
must be a \texttt{double} or \texttt{BigReal}.
%
For a vector, you need to do it for the three components.
%

If you need to reduce something new,
just add something in \texttt{ReductionMgr.h}.
\begin{lstlisting}
enum {
  ...
  REDUCTION_MY_QUANTITY,
  // semaphore (must be the last)
  REDUCTION_MAX_RESERVED
} ReductionTag;
\end{lstlisting}
%
Then in the code for slave nodes, e.g. \texttt{Sequence.C},
%
\begin{lstlisting}
  reduction->item(REDUCTION_MY_QUANTITY) += my_quantity;
\end{lstlisting}
%
and the master (e.g. \texttt{Controller.C}) can read the sum,
%
\begin{lstlisting}
  my_quantity = reduction->item(REDUCTION_MY_QUANTITY);
\end{lstlisting}


\section{\label{sec:broadcast}Broadcast}

``Broadcast'' means sending information from the master node
to the slave nodes.
%
If we want to broadcast a \texttt{Vector} called \texttt{myvec1},
we modify \texttt{Broadcasts.h}
%
\begin{lstlisting}
enum {
  ...
  myvec1Tag,
  dummyTag
};

struct ControllerBroadcasts
{
  ...
  SimpleBroadcastObject<Vector> myvec1;
  ControllerBroadcasts(const LDObjHandle *ldObjPtr = 0) :
    ...
    myvec1(myvec1Tag, ldObjPtr),
    ...
  { ; }
};

\end{lstlisting}

\end{document}
