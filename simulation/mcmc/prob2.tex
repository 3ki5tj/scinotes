\documentclass[12pt]{article}
\usepackage{amsmath}
\usepackage{tikz}
\usepackage{url}
\begin{document}


\section{Problem Set 2: Monte Carlo simulations}


\subsection{Distracted programmers}


Harry is a programmer.
%
Lately, he finds himself often distracted by
incoming E-mails during coding.
%
In the mornings,
while coding,
he checks his E-mails at a frequency of about $20$ minutes,
and it takes him about $5$ minutes to switch back to
the coding mode.
%
In the afternoons, Harry is more distracted.
%
He will check his E-mails every $10$ minutes while coding,
and it takes on average $10$ minutes
to finish reading his E-mails and to code again.

\begin{enumerate}
  \item
    Suppose Harry works four hours in the morning,
    and another four hours in the afternoon.
    How much time does he spend on coding
    in the mornings and in the afternoons?

  \item
    After talking with his Boss, Mr. Snape,
    Harry wants to improve the work efficiency
    in the afternoons.
    By exerting his will power,
    he can now resist half of the attempts of checking E-mails
    in the afternoons.
    Then how much time does he spend on coding
    in the afternoons?

  \item
    Harry's good friend Ron
    suffers from the same problem.
    %
    However,
    to improve the work efficiency in the afternoon,
    Ron decides to take a nap for $30$ minutes
    after the lunch break.
    %
    This allows him to recover the work efficiency
    back to the morning level
    for the rest of the afternoon
    (three and a half hours).
    %
    How much time does Ron spend on coding
    in the afternoons?

  \item
    Who is more efficient in the afternoons, Harry or Ron?
\end{enumerate}




\subsection{Binomial random number}



This is a programming task.
%
Using Markov-chain Monte Carlo,
generate a random number $i$ that satisfies the binomial distribution:
$$
p_i = c \, {N \choose i}
\qquad
0 \le i \le N
,
$$
where $c$ is a normalization factor.

Actually, $N$ does not have to be an integer.
Modify your code such that it may work for a non-integral $N$,
say $N = 10.5$.

Hint. We do not have to worry about $c$.
Only the ratios matter in a Monte Carlo simulation, and
$$
\frac{ p_{i+1} } { p_i } = \frac{ N - i } { i + 1 },
\qquad
\frac{ p_{i-1} } { p_i } = \frac{ i } { N - i + 1}.
$$

A sample C code is here: \url{http://ideone.com/CgL1FI}.



\subsection{Autocorrelations}

[Data from a Monte Carlo or molecular dynamcis trajectory are correlated.
  The redundancy $g$ because of the correlation
  is related to the integral autocorrelation time $\tau$
  by the following formula
  \begin{equation}
    g = 1 + 2 \, \tau.
  \label{eq:1p2T}
  \end{equation}
  This exercise is a demonstration of this formula.
]

Alice flips a coin every day.
%
If the result is a head, she writes $+1$ on her notebook;
otherwise, she writes $-1$.
%
At the end of the year, she adds the numbers up.

Bob does pretty much the same thing, except that he is lazier.
%
So every day there is two thirds chance that he does not flip the coin
and simply copies yesterday's result.
%
At the end of the year, Bob also adds the numbers up.

\begin{enumerate}
  \item
  Show that on average Bob's end-of-year number
  is roughly $\sqrt 5$ as large as Alice's
  in magnitude (no matter the sign).
  %
  In other words, Bob's numbers has
  a statistical redundancy of $g = 5$.

  \item
  Show that the autocorrelation time of Bob's numbers
  is $\tau = 2$.

  \item
  Verify Eq. \eqref{eq:1p2T}.
\end{enumerate}

Here is a demonstration:
\url{http://ideone.com/hH7U6Y}.


\end{document}
