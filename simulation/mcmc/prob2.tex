\documentclass[12pt]{article}
\usepackage{amsmath}
\usepackage[usenames,dvipsnames]{xcolor}
\usepackage{tikz}
\usepackage{url}
\usepackage{hyperref}

\hypersetup{
    colorlinks,
    linkcolor={red!30!black},
    citecolor={blue!50!black},
    urlcolor={blue!80!black}
}


\definecolor{dark-brown}{rgb}{0.2,0.1,0}

\newcommand{\hint}[1]{{\color{dark-brown}\small #1}}

\newcounter{problem}[section]
\newenvironment{problem}[1]
{
  \refstepcounter{problem}\par\bigskip
  \textbf{\large Problem~\theproblem. #1}
  \par\medskip
}
{ \medskip }

\begin{document}




\section{Problem Set 2: Monte Carlo simulations}


\begin{problem}{Distracted programmers}


Harry is a programmer.
%
Lately, he finds himself often distracted by
incoming emails during coding.
%
In the mornings,
while coding,
he checks his emails at a frequency of about $20$ minutes,
and it takes him about $5$ minutes to switch back to
the coding mode.
%
In the afternoons, Harry is more distracted.
%
He will check his emails every $10$ minutes while coding,
and it takes on average $10$ minutes
to finish reading his emails and to code again.

\begin{enumerate}
  \item
    Suppose Harry works four hours in the morning,
    and another four hours in the afternoon.
    How much time does he spend on coding
    in the mornings and in the afternoons?

  \item
    After talking with his Boss, Mr. Snape,
    Harry wants to improve the work efficiency
    in the afternoons.
    By exerting his will power,
    he can now resist half of the attempts of checking emails
    in the afternoons.
    Then how much time does he spend on coding
    in the afternoons?

  \item
    Harry's good friend Ron
    suffers from the same problem.
    %
    However,
    to improve the work efficiency in the afternoon,
    Ron decides to take a nap for $30$ minutes
    after the lunch break.
    %
    This allows him to recover the work efficiency
    back to the morning level
    for the rest of the afternoon
    (three and a half hours).
    %
    How much time does Ron spend on coding
    in the afternoons?

  \item
    Who is more efficient in the afternoons, Harry or Ron?
\end{enumerate}

\end{problem}


\begin{problem}{Binomial random number}



This is a programming task.
%
Using Markov-chain Monte Carlo,
generate a random number $i$ that satisfies the binomial distribution:
$$
p_i = c \, {N \choose i}
\qquad
0 \le i \le N
,
$$
where $c$ is a normalization factor.

Actually, $N$ does not have to be an integer.
Modify your code such that it may work for a non-integral $N$,
say $N = 10.5$.

\hint{
We do not have to worry about $c$.
Only the ratios matter in a Monte Carlo simulation, and
$$
\frac{ p_{i+1} } { p_i } = \frac{ N - i } { i + 1 },
\qquad
\frac{ p_{i-1} } { p_i } = \frac{ i } { N - i + 1}.
$$

A sample C code is here: \url{http://ideone.com/CgL1FI}.
}

\end{problem}



\begin{problem}{Autocorrelation}

[Data from a Monte Carlo or molecular dynamcis trajectory are correlated.
  The redundancy $R$ because of the correlation
  is related to the integral autocorrelation time $\tau$
  by the following formula
  \begin{equation}
    R = 1 + 2 \, \tau.
  \label{eq:1p2T}
  \end{equation}
  This exercise is a demonstration of this formula.
]

Alice flips a coin every day.
%
If the result is a head, she writes $+1$ on her notebook;
otherwise, she writes $-1$.
%
At the end of the year, she adds the numbers up.

Bob does pretty much the same thing, except that he is lazier.
%
So every day there is two thirds chance that he does not flip the coin
and simply copies yesterday's result.
%
At the end of the year, Bob also adds the numbers up.

\begin{enumerate}
  \item
  Show that on average Bob's end-of-year number
  is roughly $\sqrt 5$ as large as Alice's
  in magnitude (no matter the sign).
  %
  In other words, Bob's numbers has
  a statistical redundancy of $R = 5$.

  \item
  Show that the autocorrelation time of Bob's numbers
  is $\tau = 2$.

  \item
  Verify Eq. \eqref{eq:1p2T}.
\end{enumerate}

Here is a demonstration:
\url{http://ideone.com/hH7U6Y}.

\end{problem}


\pagebreak

\section{Problem Set 2: Monte Carlo simulations}


\begin{problem}{Distracted programmers}
\begin{enumerate}
  \item
    Four hours is $240$ minutes.
    In the mornings, the coding to emailing ratio is $4:1$.
    So Harry spends $4/5$ of the time, or $192$ minutes, on coding.
    In the afternoons, the ratio becomes $1:1$,
    so he spends on average $120$ minutes on coding.

  \item
    By resisting half of the transitions to the emailing mode,
    we have $T_{EC} = 0.5 \, T_{CE}$ (C: coding, E: emailing).
    So by detailed balance
    $$
    T_{EC} \, p_C = T_{CE} \, p_E,
    $$
    we have $p_C / p_E = T_{CE} / T_{EC} = 2$.
    This means that Harry now spends on average two thirds of the time
    on coding in the afternoons, or $160$ minutes.

  \item
    After deducting $30$ minutes of nap time,
    Ron has $210$ minutes in the afternoons.
    %
    He will spend $4/5$ of them, or $168$ minutes, or coding.

  \item
    Ron.
\end{enumerate}
\end{problem}

\begin{problem}{Binomial random number}

Here is an reference algorithm.
\begin{enumerate}
  \item
    Arbitrary set the initial $i$ between $0$ and $N$, say $i = 0$.

  \item\label{step:mcstarts}
    In each MC step, choose by half chance whether to move
    to $i - 1$ or $i + 1$.

  \item
    If the above trial move is about to
    reach a number outside $0, \dots, N$,
    reject it right away.

  \item
    Otherwise, accept the above move by probability
    $$
    p(i \to i \pm 1)
    =
    \min\left\{
      1, \frac { p_{i \pm 1} } { p_i }
    \right\}.
    $$
    where
    $$
    \frac{ p_{i+1} } { p_i } = \frac{ N - i } { i + 1 },
    \qquad
    \frac{ p_{i-1} } { p_i } = \frac{ i } { N - i + 1}.
    $$

  \item
    Repeat Step \ref{step:mcstarts} until simulation ends.
\end{enumerate}
\end{problem}

\begin{problem}{Autocorrelation}
\begin{enumerate}
  \item
    Define $N = 365$, the year-end number is given by
    $$
    S = x_1 + \cdots + x_N
    ,
    $$
    and the variance is given by
    $$
    \begin{aligned}
    \langle S^2 \rangle
    &=
    \sum_{i = 1}^N \sum_{j = 1}^N
    \langle x_i \, x_j \rangle
    \approx
    N \sum_{j = 1}^N
    c_{|i - j|}
    \\
    &\approx
    N \, ( \cdots + c_2 + c_1 + c_0 + c_1 + c_2 + \cdots )
    =
    N \, \left( 1 + 2 \sum_{k=1}^\infty c_k \right)
    .
    \end{aligned}
    $$
    In Alice's case, $c_1 = c_2 = \cdots = 0$,
    so $\langle S^2 \rangle = N$.
    In Bob's case,
    $c_1 = 2/3, c_2 = (2/3)^2, \dots$.
    So
    $c_1 + c_2 + \cdots = (2/3) / (1 - 2/3) = 2$,
    and
    $\langle S^2 \rangle = N \times 5$.

  \item
    The autocorrelation function is given by $c_k$,
    which is, in Bob's case,
    $c_0 = 1, c_1 = 2/3, c_2 = (2/3)^2, \dots$.
    The decrement of the autocorrelation function
    at successive times represents
    the rate of ``death'' of the number at time $0$.
    In other words, the probability for a number
    to persist for precisely $k$ steps is given by
    $p_k = c_k - c_{k+1} = (1/3) \cdot (2/3)^k $.
    So the average life span of Bob's numbers is given by
    \begin{equation}
    \tau \equiv \langle k \rangle
    = \sum_{k = 1}^\infty k \, p_k
    = \sum_{k = 1}^\infty \frac{ k } { 3 } \left( \frac{2}{3} \right)^k
    .
    \label{eq:tau1}
    \end{equation}
    But
    \begin{equation}
    \frac{2}{3} \, \tau
    = \sum_{k = 1}^\infty \frac{ k } { 3 } \left( \frac{2}{3} \right)^{k+1}
    = \sum_{k = 1}^\infty \frac{ k - 1} { 3 } \left( \frac{2}{3} \right)^{k}
    \label{eq:tau2}
    \end{equation}
    Subtracting Eq. \eqref{eq:tau1} by \eqref{eq:tau2}, we get
    $$
    \frac{1}{3} \, \tau
    = \sum_{k = 1}^\infty \frac{ 1} { 3 } \left( \frac{2}{3} \right)^{k}
    = \frac{ \frac{2}{9} } { 1 - \frac{2}{3} } = \frac{2}{3}
    ,
    $$
    or $\tau = 2$.

  \item
    Since $R = 5$, $\tau = 2$, we do have Eq. \eqref{eq:1p2T}.
\end{enumerate}
\end{problem}

\end{document}
