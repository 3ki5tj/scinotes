\documentclass[11pt]{article}
\usepackage[width=7.0in, height=10.0in]{geometry}
\usepackage{hyperref}
\usepackage{amsmath}



\begin{document}



\section{Addition theorem of the Legendre polynomial}



Consider two arbitrary points 1 and 2 on the unit sphere,
whose positions are described in solid angles
$\omega_1 = (\theta_1, \phi_1)$
and
$\omega_2 = (\theta_2, \phi_2)$, respectively.
%
Further define $\gamma$ as the angle between points 1 and 2.
%
The addition theorem states
\begin{equation}
  P_n(\cos\gamma)
= \sum_{m = -n}^n \frac{ 4 \pi } { 2 n + 1 }
  Y_{n, m}(\theta_1, \phi_1) \, Y_{n, m}^*(\theta_2, \phi_2).
\label{eq:addition}
\end{equation}
Below is a somewhat detailed proof of this theorem.



\subsection{Expansion in spherical harmonics}



First, we can expand the $P_n(\cos\gamma)$
as a combination of spherical harmonics $Y_{n',m}(\theta_1, \phi_1)$,
which are orthonormal.
%
One can argue that the relevant $Y_{n',m}$'s must share the same $n$,
i.e., $n = n'$.\footnote{
  Both sides of \eqref{eq:expansion1}
  satisfy the differential equation of $Y$:
\[
  \frac{ 1 } { \sin \theta }
  \frac{ \partial } { \partial \theta }
  \left(
    \sin \theta
    \,
    \frac{ \partial Y }{ \partial \theta }
  \right)
  +
  \frac { 1 } { \sin^2 \theta }
  \frac { \partial^2 Y } { \partial \phi^2 }
  + n \, (n - 1)  \, Y = 0.
\]
This would not be so,
if the right-hand side contains $n' \ne n$ terms.
}
%
So
\begin{equation}
  P_n(\cos\gamma)
=
  \sum_{m = -n}^n a_m Y_{n, m}(\theta_1, \phi_1).
  \label{eq:expansion1}
\end{equation}
%
where the coefficient $a_m$ can be computed from the integral
\begin{align}
  a_m
&=
\int d\Omega_1 \, P_n(\cos\gamma) \, Y^*_{n,m}(\theta_1, \phi_1),
\label{eq:am1}
\end{align}
where $d\Omega_1$ means
$d(-\cos\theta_1) \, d\phi_1 = \sin\theta_1 d\theta_1 \, d\phi_1$.
%
We want to show that
$a_m = 4 \pi Y_{n',m}(\theta_2, \phi_2)/ ( 2 n + 1 )$.



\subsection{The inverse expansion}



Now let us imagine a new coordination system,
in which the $z$ axis points from the origin to point 2 on unit sphere.
%
In this special coordination system,
we also have a set spherical harmonics $Y_{n', m'}(\gamma, \psi)$.
\[
  P_n(\cos\gamma)
=
\sqrt \frac { 4 \pi } { 2 n + 1 }
  Y_{n, 0}(\gamma, \psi).
\]
Thus, we can rewrite $a_m$ in Eq. \eqref{eq:am1} as
\begin{align}
a_m
&=
\sqrt \frac{4 \pi}{ 2 n + 1 }
\int d\Omega_1 \, Y_{n, 0}(\gamma, \psi) \, Y^*_{n,m}(\theta_1, \phi_1)
=
\sqrt \frac{4 \pi}{ 2 n + 1 } A_m.
\label{eq:am2}
\end{align}
%
where,
\begin{equation}
A_m
\equiv
\int d\Omega_1 \, Y_{n, 0}(\gamma, \psi) \, Y^*_{n,m}(\theta_1, \phi_1).
\label{eq:Am}
\end{equation}


In the above expression of $A_m$,
$Y_{n, 0}(\gamma, \psi)$ and $Y_{n, m}^*(\theta_1, \phi_1)$
are almost symmetric (except a complex conjugation).
%
Thus $A_m$
can be understood as the $m'=0$th expansion coefficients
of $Y_{n, m}^*(\theta_1, \phi_1)$
in terms of $Y_{n, m'}^*(\gamma, \psi)$.
%
Consider the expansion
\begin{equation}
  Y_{n, m}^*(\theta_1, \phi_1)
=
  \sum_{m' = -n}^n b_{m'} \,Y_{n, m'}^*(\gamma, \psi),
  \label{eq:expansion0}
\end{equation}
in which,
the coefficient $b_{m'}$ can be computed as
\begin{equation}
  b_{m'}
=
  \int d \Omega_{\gamma,\psi} \,
  Y_{n, m}^*(\theta_1, \phi_1) \,
  Y_{n, m'}(\gamma, \psi),
  \label{eq:bm}
\end{equation}
where, $d\Omega_{\gamma, \psi} \equiv d(-\cos\gamma) d\psi$,
and the integration over $d\Omega_{\gamma, \psi}$
and that over $d\Omega_1$.
Comparing it to \eqref{eq:Am}, we get
\begin{equation}
  b_0 = A_m.
  \label{eq:b0Am}
\end{equation}



\subsection{Determination of $b_0$}


Now our job is to determine $b_0$.
The trick is \emph{not} to do it from Eq. \eqref{eq:bm},
but to deduce it from Eq. \eqref{eq:expansion1}
by setting $\theta_1$ and $\phi_1$ to some special values
that will \emph{single out} the $m' = 0$ term.

Let us set $\theta_1 = \theta_2$ and $\phi_1 = \phi_2$.
Then the two points 1 and 2 coincide,
and $\gamma = 0$ (with $\psi$ being arbitrary).
%
We show below that only the $m' = 0$ term in the sum
of Eq. \eqref{eq:expansion1} survives.
%
This is because
\begin{equation}
  Y_{n, m'}(0, \psi)
=
  (-1)^{m'}
  \sqrt{
    \frac{ 2 n + 1 } { 4 \pi }
    \frac{ (n - m')! } { (n + m')! }
  }
  P_{n}^{m'}\bigl( \cos(0) = 1 \bigr) e^{i m' \psi},
  \label{eq:Ynm}
\end{equation}
and the associated Legendre polynomial
\begin{equation}
  P_{n}^{m'}(x)
\equiv
  \frac{ (1 - x^2)^{m'/2} } { 2^{n} n! }
  \frac{ d^{n+m'} } { d x^{n+m'} } (x^2 - 1)^{n}.
  \label{eq:Pnm}
\end{equation}
vanishes at $x = 1$, unless $m' = 0$.
Thus,
\[
Y_{n,m}^*(\theta_2, \phi_2)
=
b_0 \, Y_{n, 0}(0, \psi).
\]

The final step is to determine $Y_{n, 0}(0, \psi)$.
From Eq. \eqref{eq:Ynm}, we get
\[
  Y_{n, 0}
=
  \sqrt{
    \frac { 2 n + 1 } { 4 \pi }
  }
  P_{n}^0(1),
\]
and
from Eq. \eqref{eq:Pnm}, we get
\[
  P_{n}^{0}(1)
=
  \left.
  \frac{ 1 } { 2^{n} n! }
  \frac{ d^{n} } { d x^{n} }
  \Bigl[
    (x - 1)^{n}
    (x + 1)^{n}
  \Bigr]
  \right|_{x = 1}\
= \frac{ 1 } { 2^{n} n! }
  n! \, (1 + 1)^n
= 1.
\]
In the second step,
we have used the fact that
the $n$ differentiations much
be all applied to the factor $(x - 1)^n$
to get a non-vanishing final result at $x = 1$.
%
Thus,
$Y_{n, 0} = \sqrt{4\pi/(2n+1)}$.
and
\begin{equation}
b_0 = \sqrt{ \frac {4 \pi} { 2 n + 1 } } \, Y_{n,m}^*(\theta_2, \phi_2).
\label{eq:b0}
\end{equation}


\subsection{Finishing up}

Using Eq. \eqref{eq:b0} in Eqs. \eqref{eq:b0Am} and \eqref{eq:am2},
we get
\begin{equation}
  a_m = \frac{ 4 \pi } { 2 n + 1 }.
\end{equation}
Using this in Eq. \eqref{eq:expansion0}
proves the addition theorem Eq. \eqref{eq:addition}.




\subsection{An analogy}



The centerpiece of the above proof is the shift of coordination system.
This step may appear convoluted.
Below we give a simpler analogy
that can hopefully clear up the logic.


\end{document}
